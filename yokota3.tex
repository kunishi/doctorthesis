%#!latex main.tex

\section{Workflow Model}
\label{sec:wfmodel}

An {\em activity object}, which corresponds to a unit of work, 
is recursively defined as follows:
\begin{center}
$a=(I,O,P,S)$ where
 	$\begin{array}[t]{l}
	I=\{i_1,\cd,i_n\}\ (n\ge 1)\\
	O=\{o_1,\cd,o_m\}\ (m\ge 1)\\
	P=\mx{string}\\
	S=\mx{WFT}
	\end{array}$
\end{center}
where $I$ is a set of inputs of $a$, $O$ is a set of outputs of $a$, 
$P$ is an agent who is responsible for the execution of $a$, 
and $S$ is a WFT (defined in the following) of $a$, which executes
subworks of $a$.
Intuitively, $a$ receives $I$, $P$ executes its necessary work, and $a$
sends $O$. 
During the execution, if necessary, $a$ divides $I$, dispatches them
to $S$, monitors their execution processes, and composes $O$ from
their results. 

\begin{sloppypar}
 Strictly speaking, an activity object is defined as a quintet
 $(a,I,O,P,S)$, whose identifier is $a$. 
 In this paper, for simplicity, we denote simply $a$ or $a=(I,O,P,S)$.
 Further, when $I$, $O$, and $P$ are singletons, \{\} is omitted, if
 there is no misunderstanding. 
\end{sloppypar}

A {\em workflow template} (WFT) $W$ is defined as a set,
$\{a_1,\cd,a_n\}$, of activity objects $a_1, \cd, a_n\  (n\ge 0)$. 
Exactly, it is $(W,\{a_1,\cd,a_n\})$, the identifier of which is $W$.
For simplicity, we denote $W=\{a_1,\cd,a_n\}$, as in an activity object.

Now we can define workflows.
There are two kinds of flows in a WFT as follows:
\begin{enumerate}
\item {\em horizontal flow} ``$\lMi$''
\begin{eqnarray*}
a_1\!\lMi\!a_2&\mydefi&(O_1\supseteq I_2)\\
\{a_1,a_2,\cd,a_n\}\!\lMi\!a &\mydefi &
	(\sum_{i=1}^n O_i\supseteq I)\And
		\All j\;\Not(\bigcup_{i=1}^n O_i- O_j\supseteq I)
\end{eqnarray*}
The second definition specifies the minimality of inputs.
\item {\em vertical flow} ``$\lmi$''
\begin{eqnarray*}
a\lmi a_i&\mydefi& a_i\in S\ \mx{where}\ a=(I,O,P,S)
\end{eqnarray*}
If $a_1\lmi a_2$, $a_2$ is called a {\em child} of $a_1$ and $a_1$ is
a {\em parent} of $a_2$.
\end{enumerate}

Consider an example.
There are four activity objects, $a_1$, $a_2$, $a_3$, and $a$, such that
$O_1=\{o_1,o_2\}, O_2=\{o_2,o_3\}, O_3=\{o_3,o_1\}$, $I=\{o_1,o_2,o_3\}$.
The possible workflows are defined as follows:
\[\{a_1,a_2\}\lMi a,\ \{a_2,a_3\}\lMi a,\ \{a_3,a_1\}\lMi a\]
On the other hand, $\{a_1,a_2,a_3\}\lMi a$ is not a workflow because it 
violates the minimality condition.

To define various classes of workflows from a WFT, 
we define several restrictions:
\begin{enumerate}
\item {\em closed} WFT:\\
Consider any activity object $a=(I,O,P,S)$ in a WFT $W$.
If $a_i\in W$ for any $a_i\in S$, then $W$ is called {\em closed}.
%In this paper, we assume closed WFTs without loss of generality.

\item {\em acycle} WFT:\\
We define transitive closures of $\lMi$ and $\lmi$:
\begin{enumerate}
\item $a_1\mystack{+}{\lMi}a_2\mydefi 
	a_1\lMi a_2\Or((a_1\mystack{+}{\lMi}a_1')\And a_1'\lMi a_2)$
\item $a_1\mystack{+}{\lmi}a_2\mydefi 
	a_1\lmi a_2\Or((a_1\mystack{+}{\lmi}a_1')\And a_1'\lmi a_2)$
\end{enumerate}

If $\All a\in W.\ \Not (a\mystack{+}{\lMi}a$), 
then $W$ is called {\em acyclic} in $\lMi$, and
if $\All a\in W.\ \Not (a\mystack{+}{\lmi}a$), 
then $W$ is called {\em acyclic} in $\lmi$.
Note that a cyclic workflow corresponds to a feedback of a work.

\item {\em redundant} WFT:\\
      For any $a=(I,O,P,S)$, if $O\subseteq I$, then $a$ is called {\em
      redundant}.  A WFT containing redundant activity objects is called
      a {\em redundant} WFT.  Note that, even if two activity objects
      $a_1=(I_1,O_1,P_1,S_1)$ and $a_2=(I_2,O_2,P_2,S_2)$ have
      relations of $I_1\subseteq I_2\And O_2\subseteq O_1$, we consider
      they are not redundant if $P_1\neq P_2$.

\item {\em triangle} of WFT:\\
      Let $\leadsto=\mystack{+}{\lMi}\cup\mystack{+}{\lmi}$.
      Then if $\Ex a. (a_1\Longrightarrow a_2 \vee a_1 \longrightarrow a_2)
      \And a_1\leadsto a \And a\leadsto a_2$,
      then there exists two ways between $a_1$ and $a_2$.  A WFT
      containing $a_1$ and $a_2$ with such a relation is called {\em
      triangle}.
\end{enumerate}

{\em Workflow} is defined as a WFT $W$ satisfying these restrictions:
\[
 \forall\; a_1 \in W, \exists\; a_2 \in W.\ 
 a_1 \leadsto a_2 \vee a_2 \leadsto a_1,
\]
and $W$ constitutes a connected graph.

This definition is easily extended into a set of workflows.
Consider a set, $S$, of workflows, each of which is connected to another
workflow in $S$: that is, 
\[\begin{array}{l}
\All w_1\in S, \Ex w_2\in S.\ \\
((w_1=a\!\leadsto\!a'\And w_2=a'\!\leadsto\!a'')
\Or (w_2=a\!\leadsto\!a'\And w_1=a'\!\leadsto\!a'')),
%\All w_1,w_2\in S.\ w_1\mystack{*}{\leadsto}w_2\Or w_2\mystack{*}{\leadsto}w_1
\end{array}\]
%where $w_1\!\mystack{*}{\leadsto}\!w_2\mydefi w_1\!=\!w_2
%	\Or w_1\!\leadsto\!w_2\Or
%(\Ex w_3. w_1\!\mystack{*}{\leadsto}\!w_3\And 
%w_3\!\mystack{*}{\leadsto}\!w_2)$
and $S$ constitutes a single graph.  We extend the above definition and
call such a set of workflows a workflow generally.  The above
definitions, closedness, acyclicity, and redundancy, are also
considered in this definition of a workflow.  
In the case of triangle restriction, it is extended as follows:
$S\lMi a\And \Ex S'\subseteq S.\ S'\pq a'\And a'\pq a$.
Remark that we can
consider a set of workflows also as a set of activity objects because
flows are represented only implicitly.

Consider two workflows, $S_1$ and $S_2$.
Ordering between $S_1$ and $S_2$ is defined by subset relation: i.e.,
$S_1\subseteq S_2$.
In this order, a maximal set in a WFT called a {\em maximal} workflow.
In a maximal workflow, an activity object without any parents is
simply called a {\em parent}. 
Generally, a WFT defines multiple workflows in the sense of this definition.

A closed WFT guarantees at least one definition of a closed workflow,
which corresponds to the unit of a complete work: 
On the other hand, an unclosed workflow includes a definition of a work, 
lacking some activity objects to which a work might be submitted. 


\section{Execution Model}
\label{sec:execmodel}

To execute a WFT for a work, we must define its execution model.
The execution model of a WFT consists of two models, {\em P-box} and
{\em C-box}, each of which corresponds to horizontal and vertical flows, 
respectively.

A horizontal flow defines the following production rules:
\begin{center}
 \begin{tabular}{rl}
 If $a_1\lMi a_2$, &then $a_2\Hi a_1$\\
 If $a_1\lMi a_3$ and $a_2\lMi a_3$, &then $a_3\Hi a_1,a_2$\\
 If $a_1\lMi a_2$ and $a_1\lMi a_3$, &then $a_2\Hi a_1$ and $a_1\Hi
  a_1$\\
 If $a_1$, & then $a_1 \Leftarrow .$\\
 \end{tabular}
\end{center}
The right hand side of $\Hi$ is a set of conditions, while the left
hand side is an action.
This rule is evaluated forwardly as in ordinary production rules.
That is, by receiving all end conditions of the corresponding activity
objects, the corresponding activity object to the action is activated.
An activity object itself is represented as a fact.

A production system, as a set of such production rules, is stored in
{\bf P-box} in the WFT.

On the other hand, a vertical flow represents a set of definite clauses
as in logic programming such as Prolog.
That is, when an activity object $a$ generates children $a_1,a_2,\cd,a_n$,
the execution model is represented as follows:
\[a\hi a_1,a_2,\cd,a_n\]
Intuitively, constraints and binding information are propagated from $a$ 
to $a_1,a_2,\cd,a_n$, and if all children objects end successfully, then
$a$ ends its execution successfully.
That is, these rules are evaluated backwardly as in Prolog.

Such a set of definite clauses is stored in a {\bf C-box} in the WFT\@.
As dynamically generated child objects are also activity objects,
they are stored in P-box, not in C-box.

Consider an example (Figure \ref{f:review}).
\begin{figure}
{\small 
\begin{center}
 \setlength{\unitlength}{0.82mm}
 \begin{picture}(150,60)
\maputr{18}{50}{\shortstack[r]{all-\\submitted-\\papers}}
\put(20,49.6){\vector(2, 1){14}}
\put(20,50.4){\vector(2, 1){14}}
\put(20,49.6){\vector(2,-1){14}}
\put(20,50.4){\vector(2,-1){14}}
        \ovbx{50}{56}{32}{8}{receive}
	\put(66,56.4){\vector(1,0){14}}
	\put(66,55.6){\vector(1,0){14}}
		\maput{90}{56}{\shortstack[l]{received-\\letter}}
	\ovbx{50}{44}{32}{8}{dispatch-1}
	\put(66,43.6){\vector(1,0){14}}
	\put(66,44.4){\vector(1,0){14}}
		\maput{90}{44}{\shortstack[l]{all-\\review-\\reports}}
		\put(99,43.6){\vector(1,0){5}}
		\put(99,44.4){\vector(1,0){5}}
			\ovbx{120}{44}{32}{8}{send}
			\put(136,44.4){\vector(1,0){7}}
			\put(136,43.6){\vector(1,0){7}}
			\maput{152}{44}{\shortstack[l]{review-\\letter}}
	\vbput{49}{40}{12}
	\vtput{51}{28}{12}
	\maputr{48}{34}{\shortstack[l]{\shortstack[r]{selected-\\submitted-\\papers}}}
	\maputl{52}{34}{\shortstack[l]{selected-\\review-\\reports}}
	\ovbx{50}{24}{32}{8}{dispatch-2}
	\vbput{49}{20}{12}
	\vtput{51}{8}{12}
	\maputr{48}{14}{\shortstack[l]{submitted-\\paper}}
	\maputl{52}{14}{\shortstack[l]{review-\\report}}
	\ovbx{50}{4}{32}{8}{review}
\end{picture}

\end{center}
 }
\caption{Review Process}\label{f:review}
\end{figure}
Such a process is defined as a WFT in Figure \ref{f:wft}.
\begin{figure}
 \begin{center}
  \begin{tabular}{lcl}
   review-process & = & \{receive, dispatch-1, dispatch-2, review, send\}\\
   receive & = & (all-submitted-papers, received-letter, \\
   &   &  secretary, \{\})\\
   dispatch-1 & = & (all-submitted-papers, all-review-reports, \\
   &   & chair, \{dispatch-2\})\\
   dispatch-2 & = & (selected-submitted-papers, \\
   &   & selected-review-reports, \\
   &   & PC-member, \{review\})\\
   review & = & (submitted-paper,     review-report,     reviewer, \\
   &   & \{\})\\
   send   & = & (all-review-reports, review-letter, secretary, \\
   &   & \{\})
  \end{tabular} 
 \end{center}
 \caption{WFT of a Review Process}\label{f:wft}
\end{figure}
In a real review process, we must instantiate ``submitted-paper'',
``chair'', ``secretary'', ``PC-member'', and so on.  We will introduce
the instantiation concept of workflow in the next section.
Remark that multiple instances of a child object, ``review'', must be 
dynamically generated, after ``chair'' received ``all-submitted-papers''.

The corresponding execution model is shown in Figure \ref{f:box}.
\begin{figure}
 \begin{center}
  \begin{tabular}{ll}
   P-box:
   &receive\\
   &dispatch-1\\
   &send$\Hi$dispatch-1\\
   &$\left.
   \begin{array}{l}
    \mx{dispatch-2}_1\\
    \mx{dispatch-2}_2\\
    \vdots\\
    \mx{dispatch-2}_n\\
    \vdots
   \end{array}
   \right\}$
   \begin{tabular}{l}
    Dynamically generated\\
    activity objects
   \end{tabular}\\
   C-box:
   &dispatch-1  $\hi$dispatch-2$_1$,dispatch-2$_2$,$\cd$,dispatch-2$_n$.\\
   &dispatch-2$_1\hi$ review$_{11}$,review$_{12}$,$\cd$,review$_{1l}$\\
   &dispatch-2$_2\hi$review$_{21}$,review$_{12}$,$\cd$,review$_{2m}$\\
   &$\vdots$\\
   &dispatch-2$_n\hi$review$_{n1}$,review$_{n2}$,$\cd$,review$_{nk}$
  \end{tabular}
 \end{center}
 \caption{Execution Model of a Review Process}\label{f:box}
\end{figure}
\begin{sloppypar}
 First, three production rules, ``receive'', ``dispatch-1'', and ``send
 $\Hi$ dispatch-1'' are generated in P-box.  After activating
 ``dispatch-1'', $n$ activity objects,
 ``dispatch-2$_1$'',``dispatch-2$_2$'', $\cd$, ``dispatch-2$_n$'' are
 generated as instances of ``dispatch-2''.  Such activity objects are
 generated in P-box and its corresponding rule (definite clause),
 ``dispatch-1 $\hi$ dispatch-2$_1$, dispatch-2$_2$, $\cd$,
 dispatch-2$_n$'', which controls their execution, is generated in C-box.
 After each activity object ``dispatch-2$_i$'' is activated, its
 corresponding activity objects, ``review$_{i1}$'',$\cd$,
 ``review$_{ir}$'', for ``review'' and its rule,
 ``dispatch-2$_i\hi$review$_{i1}$,$\cd$,review$_{ir}$'', is stored in
 P-box and C-box, respectively.
\end{sloppypar}

Activity objects, generated and inserted into P-box during execution,
do not cause any conflict to existing activity objects, because they
are newly generated.
Therefore, dynamic update of P-box is an conservative extension
and does not change its semantics and does not cause new conflicts.

\section{Workflow Instance}
\label{sec:workflowinstance}

To apply WFTs defined in the previous sections to real works,
we must instantiate inputs, outputs, agents, and so on.
Such instantiated WFTs are called {\em workflow instances} (WFI)\@.
Although a WFI is an instance of a WFT, they are essentially the same.
Here we use a WFI as a WFT to execute a real work.

First we define ordering between objects and, as the results, ordering 
between WFTs.
Consider a domain $\cM$ of message objects consisting of inputs and outputs,
and a domain $\cP$ of agents.
An activity object $a=(I,O,P,S)$ is basically a function from $I$ to $O$,
defined as follows:
\begin{enumerate}\itemsep0mm\parskip0mm
\item $I\in 2^\cM$, $O\in 2^\cM$
\item $P\in 2^\cP$
\item $a\in\cA$, $S\subseteq\cA$, where $\cA$ is a set of activity objects,
defined as follows:
\begin{eqnarray*}
\cA&:=&2^\cM\times 2^\cM\times 2^\cP\times 2^\cA
\end{eqnarray*}
\end{enumerate}

Although $P$ can be defined as a function on $\cM$, we take another
domain $\cP$, because we consider that two activity objects with the
same inputs and the same outputs, but the different agents should be
defined as different objects.

For an activity object $a=(I,O,P,S)$, we consider assignment $\theta$ of
$I$ to $I'=\{i_1,i_2,\cd,i_n\}$, $O$ to $O'=\{o_1,o_2,\cd,o_m\}$, and
$P$ to $P'=\{p_1,p_2,\cd,p_k\}$.
$\cM,\cP$ are assumed to be partially ordered sets by $\spq_\cM$ and
$\spq_\cP$, respectively.
We usually omit the subscripts of $\spq$, for simplicity.
The assignment is denoted as follows:
\[\theta=
	\{I/\{i_1,\cd,i_n\},O/\{o_1,\cd,o_m\},P/\{p_1,\cd,p_k\}\},\]
which corresponds to specialization with the following relations:
\begin{eqnarray*}
I'\spq_S I&\MH&\All i\in I,\, \Ex i'\in I'.\: i'\spq i\\
O'\spq_S O&\MH&\All o\in O,\, \Ex o'\in O'.\: o'\spq o\\
P'\spq_S P&\MH&\All p\in P,\, \Ex p'\in P'.\: p'\spq p
\end{eqnarray*}
That is, $\spq_S$ is Smyth ordering, which is necessary and sufficient 
condition of being assignment.

The orderings between two activity objects $a = (I_1, O_1, P_1, S_1)$
and $a' = (I_2, O_2, P_2, S_2)$ is defined as follows:
\begin{eqnarray*}
a \spq a'  &\mydefi&I_1\spq_S I_2\And O_1\spq_S O_2\\
	   &      &\And\: P_1\spq_S P_2\And S_1\spq_S S_2.
\end{eqnarray*}
where the orderings of $S_1$ and $S_2$ is defined as:
\[
 S_1 = \{a_1, \cdots, a_n\} \sqsubseteq_S S_2 = S_1\theta \mydefi
\{a_1\theta, \cdots, a_n\theta\}.
\]
A partial assignment into $S$ is also defined as follows:
\begin{eqnarray*}
a'\spq a&\MH&\{a'\}\cup S\spq_S\{a\}\cup S
\end{eqnarray*}
An assignment $\theta$ to a WFT $W$ is defined as
$W\theta\!=\!\{a_1\theta,\cd,a_m\theta\}$.
As WFTs and WFIs are the same, we can define ordering among them by using 
assignment.

\begin{sloppypar}
 Consider the previous example (Figure \ref{f:review}).
 To instantiate ``review-process'' into ``CODAS-review'',
 we define the following assignment:
 \begin{center}
 \begin{tabular}{l}
  all-submitted-papers/\{paper$_1$,paper$_2$,$\cd$,paper$_n$\}\\
  chair/Kambayashi\\
  PC-members/\{Masunaga, Uemura, Makinouchi, Tanaka, $\cd$\}\\
  secretary/\{Takada\}\\
  $\vdots$
 \end{tabular}
 \end{center}
 Using this assignment, a WFT ``review-process'' is instantiated into
 ``CODAS-review''.  Such instantiation can be denoted as follows:
 \begin{center}
 $\begin{array}[t]{l}
	\mbox{review-process}/\mbox{CODAS-review}
 \end{array}$
 \end{center}
 We can activate a WFI by such instantiation.
\end{sloppypar}

\section{Workflow Base}
\label{sec:workflowbase}

Various information defined in the previous sections are stored in a
workflow database, that is, {\em workflow base} (WFB)\@.
For simplicity, we assume that identifiers of WFT/WFI, activity objects,
message objects and agents are global in a workflow base.

A workflow base is defined by a set of WFTs (including WFIs) and 
$(\cM,\, \spq_\cM)$, $(\cP,\, \spq_\cP)$.
Now we generalize the definitions of WFTs in Section \ref{sec:wfmodel} 
to include the identifier of the upper WFT and assignment.
That is, each WFT $W$ ($=W'\theta$) is defined as follows:
\begin{center}
\begin{tabular}{ll}
definition: &$\begin{array}[t]{l}
	   W=(W',\, \theta,\, \{a_1,\cd,a_m\})\\
	   a_1=(I_1,\, O_1,\, P_1,\, S_1)\\
	   \vdots\\
	   a_n=(I_n,\, O_n,\, P_n,\, S_n)
	   \end{array}$\\
execution model:&\begin{tabular}[t]{l}
	    P-box: $\cd$\\
	    C-box: $\cd$
	    \end{tabular}
\end{tabular}
\end{center}
Note that the execution model can be generated from the definition
initially, however dynamically generated activity objects and rules are not
stored initially.

As for a workflow base, there is an integrity constraint:
\begin{itemize}
\item Specialization hierarchy of WFTs:\ 
Ordering among WFTs are consistently defined by $(\cM,\, \spq_\cM)$ and 
$(\cP,\, \spq_\cP)$.
\end{itemize}
Furthermore, we can impose various integrity constraints defined in Section
\ref{sec:wfmodel}, according to applications' requirements.

%\end{document}
