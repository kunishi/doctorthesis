%#!latex main

\chapter{Comparisons with Related Researches}
\label{chap:relatedworks}

Many researches about workflow management systems have been done from
the various aspects.  Moreover, there are also similar concepts as
workflows in various research areas, such as groupware, process
modeling, and software process engineering.  In this chapter, we pick up
several related researches from various research areas, and compare them
with our model especially from the workflow model point of view.

\section{Groupware Approach}

Automation of office work has been one of the hot topics in the research 
area of groupware \cite{ellis:acmcs80}, and many similar concepts as
workflow management systems had been proposed before the term
``workflow'' are widely known.

\subsection{OM-1}

OM-1 \cite{ishii:tripsj86,ishii:jip91}, categorized as an {\em office
procedure model}, represents knowledge of well-structured cooperative
work both in control flow and in office structures, such as data
structures and organization structures, using object-oriented modeling.
Three objects --- data, activity, and agent --- are provided as basic
office objects.  Office procedure is represented as a graph whose nodes
are basic office objects.  Links between office objects represent
various relationships between office objects, such as control flow,
responsibility, belonging, and operations on data.  Four styles of
connections for control flows are provided:  AND join, AND fork, XOR
join, and XOR fork.  

OM-1 has the ability to represent control flow and office structures in
a uniform style.  However, this ability makes office procedures to be
much complex.  Our workflow model manages control flows and office
structures in a separate way but with relationships.  This makes
workflows to be simple.  Moreover, our model has a formal execution
model which OM-1 does not provide.

\subsection{Action Workflow}

Action Workflow \cite{medina-mora:cscw92} is a workflow model based on
conversation act theory \cite{winograd:book86}.  It represents all
cooperative works as an action workflow loop, a loop of four processes:
proposal, agreement, performance, and satisfaction.  Each process in an
action workflow loop can be an action workflow loop.  Hence an action
workflow is organized hierarchically in general.  This paper also gives
an architecture of workflow management systems based on action workflow
model.  The Workflow Management Server, a core system of the workflow
management system, manages both the definitions of workflows and the
progress of workflows using databases.
% STF (Standard Transaction Format) processors ?

Action Workflow is a model suitable for ill-structured cooperative work.
However, processes provided in Action Workflow are too primitive to
represent well-structured cooperative work in a simple way.  Moreover,
the execution semantics of the workflows in Action Workflow is not
clear.  Data models of the workflows in the Workflow Management Server
is not discussed.

\subsection{Regatta}

\begin{sloppypar}
 The Regatta project, formed in 1991 to develop software to support
 workgroups and to aid in reengineering work processes, proposes a model
 for collaborative work and a graphical language to support this model 
 \cite{swenson:coocs93}.   In this model, work process is modeled as a
 network of tasks, each of which represents a task request, commitment,
 or question.  Though this modeling concept is similar with Action
 Workflow,  the model provides more functions than Action Workflow such
 as dynamic modification of flows, incremental automation, etc.
\end{sloppypar}

The workflow model of the Regatta project has the similar concepts as
WFT/WFI\@.  Before workflow instances are executed, they are created from
workflow templates by assigning real-world entities into variables.  The
features such as hierarchical workflows, dynamic changes, view functions
as private to-do lists, etc., are also similar as those of our models.
However, formal semantics is not given in these models.  Moreover, the
view functions in these models are somewhat ad-hoc, not based on formal
operations.

\subsection{MEGUMI}

Tarumi et at. proposed and developed a workflow management system based
on MEGUMI, rule based e-mail system \cite{tarumi:tipsj95,%
yoshifu:tripsj95-1}.  In this system, workflows are realized by the
formed e-mail circulation with the rule-based control in MEGUMI\@.  Hence
this system does not assume a central workflow server.  The rules
realizing workflows are described as ECA rules by HyperScheme, a
programming language extending Scheme.  They also discussed about
databases for workflow management \cite{yoshifu:tripsj95-1}.

Problem of the system is the capability of HyperScheme from the workflow
definition point of view.  As HyperScheme is a programming language for
general purpose, it can represent illegal workflows.  Though GUI based
workflow definition tool is provided in order to restrict the ranges of
workflows to be defined, there still remains possibilities users define
illegal workflows by customizing rules not using the workflow definition
tool.

\subsection{WorkWeb}

WorkWeb \cite{tarumi:tripsj95-6, tarumi:tripsj96-1} is much
different from other workflow management systems.  It deals with some
constraints about the common resources of the organizations --- objects,
persons, and money.  Agents are provided for each person, each resources,
each shared data, and each workflows, and they negotiates each other to
resolve conflicts of these constraints.  The negotiation is done over
a number of work processes running in parallel at one organization.

WorkWeb can treat more wider constraints than our workflow model shown
in Chapter \ref{chap:ext}.  However, the way for managing each workflow
process is not discussed in WorkWeb.  This means the constraint
management in WorkWeb can be compatible with our workflow model.

\section{Process Modeling Approach}

Process modeling is a research area closely related with workflows.  It
has been studied mainly in software process modeling, in modeling of
reactive systems, and in office information systems.  In recent years,
the similarities between process modeling and workflows are pointed out
\cite{robinson:nsf96}, and researches of applying process modeling into
workflows have been studied.

\subsection{Activity Management System}

\begin{sloppypar}
 Activity Management System (AMS) \cite{tueni:studiescscw91} is a
 knowledge-based system which supports the representation and execution
 of procedural knowledge, which is expressed like as workflows.
 In this system, the architecture of AMS can be stratified into three
 layers: a control structure to handle the interruption, resumption, and
 cancelation of tasks; a mechanism to schedule tasks that are in
 progress; a mechanism to handle missing information.  This system also
 provides some office operators such as send, request, acknowledge, and
 answer.  Concatenating these operators, it organizes procedural
 knowledge which is similar with speech act model \cite{winograd:book86}.
 The operators are implemented based on Petri nets.
\end{sloppypar}

Though AMS can express cooperative works flexibly, there is no
discussion about agents as executors of the units of work.  

\subsection{Statechart}

Statechart \cite{harel:scp87} is a hierarchical finite automaton with
concurrent execution and broadcast mechanisms for communications between
them.  Its semantics is formerly defined like finite automaton using
transitions between sets of exclusive states, called configurations.
Based on statechart, a computerized environment for the development of
reactive systems, called STATEMATE \cite{harel:ieeese90}, is proposed.
STATEMATE has a design database of statechart with general query
language, which have an expressive power almost same as that of the
conjunctive query in relational databases.  Though STATEMATE is proposed 
for modeling reactive systems, it is also suitable for process
modeling.  From this viewpoint, we can find some papers applying
statecharts into software process modeling \cite{kellner:hawaiiconf89,
humphrey:icse89}.

Many similarities can be found between our workflow model and
statechart/STATEMATE\@.  First, both are based on finite automaton with
some extensions like hierarchy, concurrent execution, synchronization,
etc.  Secondly, both give formal semantics of the models.  Thirdly, both 
systems use database for process data management.  And finally, both
systems provide general query languages based on relational algebra.
The query language of STATEMATE can express queries on a set of states
as well as on a set of statecharts, although few discussions about the
data model of statecharts are found in \cite{harel:ieeese90}.

In spite of the similarities with our model, statecharts lacks some
features which are important for workflow management.  Statechart
provides no function to change its own execution dynamically; there is no
discussion about agents as executors of work, instantiation like
WFT/WFI, integrity constraints of the model, etc.

\subsection{KyotoDB}

KyotoDB \cite{matsumoto:asst90} is a software project database based on
an object model.  In this system, both process programs and data are
encapsulated into objects, and stored into one database.  Work process,
called unit workload in KyotoDB, is modeled as a sequence of work slice,
a directed graph of work process nodes, validation nodes, and
communication nodes with other unit workloads.

KyotoDB treats the units of work as an object in a strict sense; each
object permits to be accessed only via its own methods.  When the user
wants to access objects based on set operations, he has to use general
query languages on OODBMS such as OQL.

\section{Database Approach}

As mentioned in Section \ref{sec:pre:wfms}, almost all researches about
workflows from the database point of view treat transaction management
among processes.  We found few papers discussing workflows from the data
model point of view.

\subsection{C\&Co}

C\&Co \cite{forst:jdps95} is a programming language with some
coordination extension features in C\@.  Concurrent execution of
processes, dependencies between processes, transactions with retrying or
compensating options, synchronization, etc. are newly introduced in C.

In order to implement workflow management systems with database
technology, programming language should have these features newly
introduced in C\&Co.  However, in the workflow language point of view,
C\&Co does not provide sufficient capabilities for representing
workflows; that is, the functions C\&Co provide for workflow management
are too primitive to represent workflows.

\subsection{Event-Condition-Message Rules}

M. Rusinkiewicz et al. propose a rule-based workflow model closely
related with database technologies \cite{jean:codas96}.  In this model,
a workflow is represented as a set of Event-Condition-Message (ECM)
rules, whose semantics is ``When event $E$ occurs, evaluate condition
$C$.  If it evaluates to true, send message $M$''.  Tasks are implicitly
related to each other through ECM rules; that is, when a task is
finished, an event of ``finishing'' is sent to a workflow and the
corresponding ECM rules are fired and evaluated.  If a condition
evaluates to true, a message is sent to a set of tasks to be invoked.  A
workflow in this model is depicted as a hierarchical state transition
diagram.  They also discusses database schema to store workflow
definitions and workflow instances, query languages, and system
architecture of prototype.

Their standpoints are quite similar as ours: they use database
management systems to store the static specification of workflows, as
well as their run-time information; they provide a method for querying
the database to get several information about workflows.  However,
workflows are stored as a directed graph in their model.  This makes
editing a workflow specification by database query languages to be
difficult, as they pointed out in \cite{jean:codas96}.