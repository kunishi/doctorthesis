%#!latex main

% \documentstyle[math,eepic,eepicemu]{jarticle}
%\documentstyle[math,eepic]{jarticle}
%\renewcommand{\baselinestretch}{1.2}
%\renewcommand{\labelenumi}{\theenumi)}
%\begin{document}
%\large

% 2.3
\begin{sloppypar}
 \section{Definite Clauses, Production Rules, and ECA Rules}
\end{sloppypar}

In this thesis, we use definite clause based logic programming such as
pure Prolog, production rules, and ECA (Event-Control-Action) rules to
control workflows.  We assume that the readers are familiar with them.
Here we introduce their basic concepts, definitions, and behaviors.

\subsection{Definite Clauses and Logic Programming}

Any closed formula in the first order logic can be transformed into a
set of clauses, each of which is in the following form:
\[p_1\Or p_2\Or\cd\Or p_n\Or \Not q_1\Or \Not q_2\Or\cd \Or\Not q_m\]
where $p_i$, $q_j$ ($0\le i\le n$, $0\le j\le m$) are atomic formulas (atoms).

A clause with at most one positive atom is called a {\em clause},
written as follows:
\[\begin{array}{l}
p\hi q_1,q_2,\cd,q_m.\\
p\hi.\\
\hi q_1,q_2,\cd,q_m
\end{array}\]
where $p$ is a positive atom and $q_i$ is a negative atom.  The
first and the second clauses are called {\em definite clauses}, a
set of which is called a {\em program} or a {\em database}.  The third
clause is called a {\em goal}.  Without loss of generality, we can
assume that the second clause does not have any variables, as in
extensional databases of deductive databases.  The second one is
called a {\em fact}, and the first one is called a {\em rule}.  The
left hand side of $\hi$ is a {\em head} and the right hand side of
$\hi$ is a {\em body}.

As usual, we can define three kinds of formal semantics: declarative
semantics as the minimum Herbrand model, procedure semantics such as
SLD resolution, and least fixpoint semantics.  Here we focus on the
procedure semantics.  Given a query ?-$q$ to a program $P$, query
processing is represented as the following sequence of pairs of a set
of goals and a set of substitutions:
\[(G_0,\emptyset)\mi(G_1,S_1)\mi\cd\mi(G_i,S_i)\mi\cd\mi(\emptyset,\S_n)\]
where $G_0=\{q\}$, and if 
$p\hi p_1,p_2,\cd,p_m\in P$, $q'\in G_i$, and $p\theta=q'\theta$, then
\[\begin{array}{l}
G_{i+1}=G_i\cup\{p_1\theta,p_2\theta,\cd,p_m\theta\}\\
S_{i+1}=S_i\cup\theta
\end{array}.\]

To control workflows, we treat an atom corresponding to a work: i.e., a rule
$p\hi p_1,p_2,\cd,p_n$ means that works $p_1,p_2,\cd,p_n$ should be
done to complete a work $p$.  In other word, if $p$ is activated by
and receives inputs in some arguments as a result of unification, $p$
activates subgoals $p_1,p_2,\cd,p_n$ and sends inputs to them as a
substitution.  If subgoals are executed successfully, they return
outputs as new substitution.  If a subgoal is defined by another rule,
it activates the corresponding rule. 

\subsection{Production Rules}

A production rule is a basic component of an expert system and 
defined as 
\begin{center}
if {\em condition-part}, then {\em action-part},
\end{center}
where the condition-part consists of multiple conditions.
Here, we denote a production rule as $w\Hi w_1,w_2,\cd,w_n$.
In this thesis, we consider each condition as the completion of its 
corresponding work, and an action as a newly activated work.

Differently from definite clauses, we evaluate production rules
forwardly as one way information passing from condition-part to
action-part as usual.

\subsection{ECA Rules}

We introduce an ECA (event-control-action) as an extension of a production
rule for efficient processing.
The semantics of a ECA rule is that if an event E occurs, then a
condition C is evaluated, and if C is satisfied, then an action A is
executed.

Comparative with a production rule, an ECA rule has some advantages
for our application, workflow:
\begin{itemize}\itemsep0mm\parskip0mm
\item In workflow, it is easier to model events such as feedback and
analyze of failure.
\item We can classify rules according to kinds of events and
optimize them.
\item An event corresponds to the timing of evaluation, while
conditions correspond to the contents of evaluation.
\end{itemize}
 Recently ECA rules are used in the context of active
databases \cite{mccarthy:sigmod89}, while
we use them to classify kinds of flows among works.

% 2.4
\section{Agent as an Executer}

A workflow defines a set of works and their structured flows, where
each work may be executed either automatically by a program, or by a person:
i.e., simply speaking, the executer of work can be abstracted as a 
{\em problem solver}
or an {\em agent}.  In this thesis, we
use a {\em problem solver} as a general term for
a database system, a knowledge-base system, a constraint solver, an
expert system, an application program, and a person,
and we employ a concept {\em agent},
proposed by a heterogeneous distributed cooperative problem solver,
{\em Helios}\cite{helios94,helios95}, as an abstracted problem solver
with the same protocol.

A basic concept (in {\em Helios}) is an {\em agent}, defined as follows:
\centab{lcl}
agent	&:=	&(capsule, problem-solver)\\
	&$|$	&(capsule, environment, \{agent$_1$, $\cd$, agent$_n$\})
\tabcen
A {\em simple} agent is defined as a pair of a {\em capsule} and a
problem solver: intuitively, a problem solver is wrapped with a
capsule as in Figure \ref{f:agent}.
A {\em complex} agent is defined as a triple of a capsule, an {\em
environment}, and a set of agents (agent$_1$,$\cd$,agent$_n$), where
an environment is a field where agent$_1$,$\cd$,agent$_n$ can exist
and communicate with each other.
Intuitively, as a pair of an environment and a set of agents can be
considered also as a problem solver, a new agent can be defined by
wrapping them by a capsule.
That is, an agent can be also hierarchically organized.
Figure \ref{f:agent} shows such structures.

\begin{figure}
\bpics{.7}{166}{60}(0,2)\thinlines
\maput{83}{57}{user (environment)}\vtput{83}{49}{5}\vbput{83}{49}{4}
\maput{20}{47}{complex agent}
{\Thicklines\ovbx{83}{20}{166}{48}{}\maput{83}{40}{capsule}}
\ovbx{83}{17}{158}{38}{}\maput{45}{30}{environment}

\maput{33}{21}{simple agent}
{\Thicklines\ovbx{33}{9}{46}{18}{}\maput{33}{14}{capsule}}
\ovbx{33}{6}{38}{8}{problem solver}

\maput{82}{31}{complex agent}
{\Thicklines\ovbx{107}{14}{90}{28}{}\maput{107}{24}{capsule}}
\ovbx{107}{11}{82}{18}{}\maput{107}{16}{environment}
{\Thicklines\ovbx{88}{9}{34}{6}{agent}
	    \ovbx{126}{9}{34}{6}{agent}}
\epic
\caption{Basic Model of Agent}\label{f:agent}
\end{figure}

A capsule and an environment are defined as follows:
\begin{center}
\begin{tabular}{lll}
 capsule	   & := & (agent-name, methods, self-model, \\
 & &  negotiation-strategy)\\
 environment & := & (agent-names, common-type-system, \\
 & &  negotiation-protocol, ontology)
\end{tabular}
\end{center}
An {agent name} in a capsule is an identifier of the corresponding
agent, and {\em agent names} in an environment specify what agents
exist in the environment.
{\em Methods} in a capsule define {\em import} and {\em export}
method protocols of the corresponding agent.
An agent with only import methods is called {\em passive} and an agent
with both methods is called {\em active}: that is, only an agent which
sends new messages through export methods can negotiate with other
agents.
A {\em common type system} in an environment enforces all agents under
the environment to type all messages strongly.
A {\em self model} in a capsule defines what the agent can do.
An environment extracts necessary information from self models in
agents to dispatch messages among agents.
Under a {\em negotiation protocol} in an environment, each agent
defines a {\em negotiation strategy} to communicate with other agents.
An {\em ontology} defines the transformation of the contents of
messages among agents, while a capsule converts the syntax and type of
messages between the common type system and the intrinsic type system
of the corresponding problem solver.
These information is defined in {\em CAPL} (CAPsule Language) and
{\em ENVL} (ENVironment Language).

Although various information is defined in each environment and each
agent, a {\em message} among agents is in the form of a global {\em
communication protocol} consisting of the message identifier, the
identifier of a sender agent, the identifier of a receiver agent, a
transaction identifier, and a message.
A message identifier is common in a query message and answer messages.
A transaction identifier is used to identify a negotiation process as
a transaction, which can be nested.

A user can play three roles in {\em Helios}: an end user, an outermost
environment, and a problem solver if he keep the above same protocol.

For communication between a user and an agent, a user can give his
user model, which corresponds to a common type system and data
structures defined in an outermost capsule.
Given a user model to an agent, its capsule transforms all messages
between the user and the agent.

A user is defined as the outermost environment where there is only one
(simple or complex) agent.
If an internal agent cannot solve a problem, the problem is thrown out
in the outer environment.
Hence, a user receives unsolvable problems finally.
If the user returns the answer to the agent, the agent continues to
process the suspended message.

Furthermore, a user may be defined also as an agent, that is, a user
can process a message sent by its capsule and return its result to
the capsule.
This feature helps not only prototyping a system, but also
constructing a groupware environment, if multiple users are defined
as agents.
In this thesis, we use such features of an agent as an executer of
a work.
Such models make prototyping multi-agent programming in our model
easier.

Relations among users and agents are shown in Figure \ref{f:user}.
\begin{figure}
\bpic{88}{68}\Thicklines
\ovbx{4}{34}{8}{68}{\shortstack[c]{U\\S\\E\\R}}
\path(38,61.5)(8,61.5)\path(38,62.5)(8,62.5)
%\path(14,2.5)(8,2.5)\path(14,3.5)(8,3.5)
\path(14,5.5)(8,5.5)\path(14,4.5)(8,4.5)
\ovbx{52}{62}{28}{12}{environment}\path(52,56)(52,52)
\ovbx{52}{46}{20}{12}{capsule}\path(52,40)(52,36)
\ovbx{52}{30}{28}{12}{environment}\path(26,18)(52,24)(50,14)\path(52,24)(78,14)
\ovbx{26}{15}{20}{6}{capsule}\path(26,12)(26,10)
\ovbx{26}{5}{24}{10}{\shortstack[c]{problem\\solver}}
\ovbx{50}{7}{20}{14}{agent}
\maput{64}{7}{$\cd$}
\ovbx{78}{7}{20}{14}{agent}
\epic
\caption{User as an Environment and an Agent}\label{f:user}
\end{figure}

%\begin{thebibliography}{9}
%\bibitem{helios94}
%Kazumasa Yokota and Akira Aiba, 
%``A New Framework of Very Large Knowledge Bases'', 
%{\it Knowledge Building and Knowledge Sharing},
%eds. by K. Fuchi and T. Yokoi, pp.192-199, Ohmsha and IOS Press, 1994.
%
%\bibitem{helios95}
%Akira Aiba, Kazumasa Yokota, and Hiroshi Tsuda,
%``Heterogeneous Distributed Cooperative Problem Solving System
%Helios and Its Cooperation Mechanisms'',
%{\it Int. Journal of Cooperative Information Systems}, Vol.4, no.4, 
%pp.369-385, Dec., 1995.
%\end{thebibliography}
%\end{document}