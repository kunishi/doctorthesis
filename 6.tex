%#!latex main

% \documentstyle[math,eepic,eepicemu]{jarticle}
%\documentstyle[math,eepic,epsf]{jarticle}
%\renewcommand{\baselinestretch}{1.2}
%\renewcommand{\labelenumi}{\theenumi)}
%\begin{document}
%\large

% Agent Reconsidered
\chapter{Agents Reconsidered}
\label{chap:agentReconsidered}

To make a collaborative work possible, related programs and persons
should have a common protocol, which resolve each heterogeneity with
some protocol.  We call these entities related with the collaborative
work such as database systems, knowledge-base systems, constraint
solvers, application programs, persons, and so on, as a problem solver.
Therefore {\em agents}, wrapping problem solvers by a common protocol,
should be considered in a collaborative work.

Considering a big problem such as workflow, 
we can find many applications which
require multiple {\em heterogeneous} problem solvers:
\begin{itemize}\itemsep0mm
\item Modeling Heterogeneity\\
The complexity of a given problem requires a combination of multiple
heterogeneous problem solvers,
\item Spatial Heterogeneity:\\
Spatially distributed problem solvers are required to process a given
problem.
\item Temporal Heterogeneity:\\
For a new problem, a new problem solver is not necessarily developed:
that is, multiple existing problem solvers must be reused for a new
problem.
\end{itemize}
There have been some approaches: an arithmetic calculator in Prolog
and a constraint logic programming language with a single constraint
solver.  Such a restricted approach seems to be neither flexible nor
promising for most applications.

Further, considering the spread of distributed environments, there
might be similar sources, each of which does not have complete
information.  
In such an environment, we can frequently get better results by
accessing and merging multiple information sources or multiple
problem solvers.
In other words, cooperation among distributed sources are frequently
required.
Considering such applications and environments, heterogeneous,
distributed, cooperative problem solvers will become more important
and can play a role of a very large knowledge-base.
The situation is almost the same, because there might be many agents
who can possibly solve a given problem.
A coordinator agent can select and judge an agent for the work.

From workflow and workflow base points of view, in a unit of workflow
definition, all agents should have the same protocol, for which we
use an environment concept as in Helios.
In this chapter, we discuss how to define such environments 
for workflow management.

\section{Logic-Oriented Definition}

In the previous chapters, we consider a workflow as a set of
activity objects  by corresponding an activity object to a work.
An agent which (or who) is responsible for a work is embedded in its activity
object.
However, as an agent usually has multiple kinds of works, we consider
another definition of an agent:
\[a/[w(I)=O]\Hi S|C\]
where $a$ is an agent identity, $w$ is an work identity, $I$ is a
set of inputs, $O$ is a set of outputs, $S$ is a set of sub-agents, which
are called by $a$, and $C$ is a set of constraints which are delivered
from $a$ to $S$.
As in deductive object-oriented languages such as F-logic\cite{f-logic}, 
we can correspond the rule to conventional object-orientation concepts: 
a work identity corresponds to a method identifier, $I$ is a set of
input values, $O$ is a set of return values, and $S|C$ is its 
implementation.

As an agent do many kinds of works, it is defined as follows:
\[\begin{array}{l}
a/[w_1(I_1)=O_1]\Hi S_1|C_1\\
a/[w_2(I_2)=O_2]\Hi S_2|C_2\\
\mn\hs{1cm}\vdots\\
a/[w_n(I_n)=O_n]\Hi S_n|C_n
\end{array}\]
or
\[\begin{array}{l}
a/[w_1(I_1)=O_1, w_2(I_2)=O_2,\cd w_n(I_n)=O_n]\\
\Hi S_1\cup S_2\cup\cd\cup S_n|C_1\cup C_2\cup\cd\cup C_n
\end{array}\]
In this definition, we introduce concurrency in $a$ easily because
each work does not share variables with other works, however it is very 
difficult to control it.
Further, as each agent is not encapsulated, it is difficult to correspond
an agent to a person who is responsible for its work.

\section{Agent-Based Definition}

Considering autonomous agents who execute various works,
we had better introduce a concept of an {\em agent}, which can
be defined independently from definitions of workflows.
Each agent can estimate workload and schedule a set of works.
In this subsection, we define an agent in the sense of {\em Helios}
and discuss an agent as a person.

\subsection{Agents and Environments}

\begin{sloppypar}
 As mentioned in Section 2.4, an agent is defined as follows:
 \begin{center}
 \begin{tabular}{lcl}
 agent	   &:=& (capsule, problem-solver) $|$\\
           &  & (capsule, environment, \{agent$_1$, $\ldots$, agent$_n$\})
 \end{tabular}
 \end{center}
 where a {\em capsule} is a module which contains a translator, and each
 problem-solver is enclosed in a capsule.  An encapsulated problem
 solver is called an {\em agent}, and a problem-solver is called a {\em
 substance}.  
\end{sloppypar}

A {\em simple} agent is defined as a pair of a {\em capsule} and a
problem-solver: conceptually, the problem-solver is encapsulated in
the capsule.

A {\em complex} agent is defined as a triple of a capsule, an
environment, and a set of agents (agent$_1$,$\ldots$,agent$_n$), where
an environment is a field where agent$_1$,$\ldots$,agent$_n$ can exist
and communicate mutually.  Conceptually, a pair of an environment and a
set of agents can be considered as a problem-solver; a new agent can be
defined by encapsulating them.  That is, an agent can be hierarchically
organized.  We show such structure in Figure 2.2.  Since an encapsulated
problem-solver or an environment can be considered as an agent and the
outside of an agent is an environment, the user can be considered as an
environment.

A common space for agents is called an {\em environment}.  An
environment takes care of message-passing between agents in it, and
manages global information for those agents. 
Each agent has its own logical name that is unique in the
environment.

A capsule and an environment is defined as follows:
\begin{center}
\begin{tabular}{lll}
capsule	   &:=&(agent-name, methods, self-model,\\
           &  &\ translation-rules, negotiation-strategy)\\
environment&:=&(agent-names, common-type-system,\\
           &  &\ negotiation-protocol, ontology)
\end{tabular}
\end{center}

In a capsule, an {\em agent-name} is an identifier of the agent.  Each
agent has its own agent-name that is unique in the environment.  A
{\em method} is a definition of {\em import methods} and {\em export
methods}. An import method defines a method by which the agent is
called, and an export method defines a method that the agent can call.
An agent with only import methods is called {\em passive} and an agent
with both methods is called {\em active}: only an agent which sends
new messages through export methods can negotiate with other agents.
A {\em self model} defines what the agent can do in terms of the name
of functions provided by the agent.  An environment extracts the
necessary information from self models in agents to dispatch messages
between agents.  {\em Translation-rules} define translation between
internal representation of the agent and common representation given
by its environment in the initialization of the problem solving
system.

In an environment, {\em agent names} state which agents are in the
environment.  A {\em common type system} defines a type system used to
type all messages in the environment.  A {\em negotiation protocol}
defines the protocol used by all agents in the environment.  Under a
{\em negotiation protocol} in an environment, each agent defines a
{\em negotiation strategy} to communicate with other agents.  An {\em
ontology} defines the transformation of the contents of messages
between agents, while a capsule converts the syntax and type of
messages between the common type system and the intrinsic type system
of the corresponding problem-solver.

To define this information, we introduce a capsule description
language {\em CAPL} (CAPsule description Language), and an environment
description language {\em ENVL} (ENVironment description Language). 
Programs written in those languages are processed by their
corresponding compilers.

Although various information is defined locally in each environment
and each agent, a {\em message} between agents is in the form of a
global {\em communication protocol} consisting of the following:

\begin{itemize}\itemsep0mm\parskip0mm
\item Message identifier\\
An identifier used for identifying a message. This field is unique
within an environment.

\item Message type\\
As described above, a message is either a method
invocation or an answer. The former message is called a {\em query
message}, and the latter message is called a {\em reply message}. This
field is used to distinguish a query message from a reply message.

\item Sender agent identifier\\
This field contains the agent name of the agent that sends this
message.

\item Designation of destination agents\\
The methods of designating destination agents in a query message are
described below. In a reply message, this
field contains the agent name of the agent that is the sender of the
corresponding query message.

\item Transaction identifier\\
If the update of the content of a destination problem-solver is
attendant on the invocation of a message, then a transaction
identifier is required to control it.  This field contains a
transaction identifier. For nested transactions, a transaction
identifier with a nested structure is used.

\item Status\\
This field contains information on the status of invoked methods for
error handling.

\item Message content\\
In a query message, this field contains a method invocation, and in a
reply message, this field contains the answer to the invocation.
\end{itemize}

In the previous subsection, we consider some difficulties in logic-oriented
approach.
On the other hand, in this agent-oriented approach, as each agent is
autonomous, concurrency control 
can be done by a capsule or a substance, and correspondence with a person
is made possible by wrapping by a capsule.

\subsection{Coordination among Agents}

In our workflow model, $(I,O,P,S)$, an agent $P$ controls a set,
$S=\{a_1,a_2,\cd,a_m\}$, of activity objects.
It can be written as a complex agent as follows:
\[P=(c,e,\{a_1,a_2,\cd,a_n\})\]

In Helios, messages are dispatched as follows:
\begin{itemize}\itemsep0mm
\item Initialization\\
First, during the initialization of agent processes in the
environment, the environment constructs a map of a logical agent name
and a physical process address (or IP address).
Secondly, the environment gathers method information and function
information in self models from each agent and constructs two kinds of
maps: a method and an agent name; a function name and an agent name.
Such maps work for dispatching messages among agents.

\item Dispatching:\\
As a method or a function does not necessarily corresponds to an agent
uniquely, a message is possibly sent to multiple agents.
This mechanism is useful for the followings:
\begin{itemize}\itemsep0mm
\item It is unnecessary to specify an agent name in problem solvers
explicitly.
\item It is possible to send simultaneously a message to possible
agents.
\end{itemize}
An environment decides to send a message sequentially or in parallel to
candidates listed by the maps, and processes answers sequentially or
by grouping as a set.
In the case of set grouping, aggregation functions can be specified in
an environment.
Such a mode can be selected in a query message.
\end{itemize}

As already mentioned, differently from Helios, $P$ is responsible for
dispatching works (messages) to agents and coordinating their execution.
In the sense, $P$ provides the capsule $c$ and includes the functions
of an environment $e$.

A message is analyzed and its corresponding processing plan is 
constructed as a dependency graph by a parent agent as in 
a query in conventional distributed databases.
Synchronization information between sub-messages is attached to each
sub-message and controlled by the capsule of each agent: i.e.,if necessary, 
child agents can communicate with each other.
The coordination protocol between a parent agent and child agents
is rather simple:
\begin{itemize}\itemsep0mm
\item success/failure:\\
An child agent sends a success or failure message to the parent agent.
The parent agent redispatches the work to another child agent or cancels
all works dispatched to all child agents related to the work.
There might be various reasons of failures: unsatisfaction of time
constraints, lack of resources, and so on.

\item resource requirements:\\
A child agent can request some resources necessary for executing its
own work to the parent agent.
The parent agent sends additional resources to the child agent if
the parent has enough resources.
If the parent does not have necessary resources, it requests them 
to other child agents.
If a child agent has enough ones, it sends them to the parent.
\end{itemize}

We assume that a person involved in some workflow has an appropriate
interface to the capsule of the corresponding agent.
As a person can do many works, he is connected to many capsules.
In this thesis, we do not discuss the topic.

\subsection{Relations between Workflow Base and Agent Definitions}

We have two kinds of definitions about a workflow: a workflow
definition and a agent definition.
As mention in Section 3.5, a workflow base consists of a workflow 
definition and its execution model.
A workflow definition corresponds to a static aspect of workflow,
while an execution model corresponds to its dynamic aspect, where
activity objects are dynamically generated and activated in P-box and their
relations are defined in C-box in the form of definite clauses.
A definite clause, $p\hi q_1,q_2,\cd,q_n$, is a dispatching plan,
which corresponds to an activity object $p$, which is also in P-box.
That is, an isolated activity object in P-box just corresponds to
an agent definition.

\section{Summary}

Considering a big problem such as workflow, 
we can find many applications which
require multiple {\em heterogeneous} problem solvers.
To make a collaborative work possible, related programs and persons
should have a common protocol, which resolve each heterogeneity with some
protocol.

There have been some approaches: an arithmetic calculator in Prolog
and a constraint logic programming language with a single constraint
solver.  Such a restricted approach seems to be neither flexible nor
promising for most applications.

From workflow and workflow base points of view, in a unit of workflow
definition, all agents should have the same protocol, for which we
use an environment concept as in Helios.
In this chapter, we discussed how to define such environments 
for workflow management.

A problem solver, a generic term of database systems, knowledge-base
systems, constraint solvers, application programs, persons, and so on,
is called an {\em agent} by wrapped by a common protocol.

Considering such applications and environments, heterogeneous,
distributed, cooperative problem solvers will become more important
and can play a role of a very large knowledge-base.
The situation is almost the same, because there might be many agents
who can possibly solve a given problem.
A coordinator agent can select and judge an agent for the work.

%\begin{thebibliography}{9}
%\bibitem{f-logic}
%M.~Kifer and G.~Lausen, ``F-Logic --- A Higher Order Language for
%Reasoning about Objects, Inheritance, and Schema'', {\it Proc. 
%{\small\it ACM SIGMOD} Int. Conf. on Management of Data}, 
%pp.134-146, Portland, June, 1989.
%\end{thebibliography}
%\end{document}