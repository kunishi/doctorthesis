%#!latex main

\chapter*{Abstract}
\addcontentsline{toc}{chapter}{Abstract}
\markboth{Abstract}{Abstract}

Workflow management system (WFMS), one of the groupware supporting
asynchronous distributed interaction, becomes remarkable not only in
practical business area but also in research area.  

WFMSs mainly support well-structured collaborative works using
explicitly defined flows of works (workflows).  They controls the
invocation order in workflows automatically, and manages several
resources of the work.

As a number of workflows run concurrently using shared resources of
organizations, transaction management with concurrency control is an
important technology for WFMSs.  In this sense database technologies are
indispensable for the infrastructure of WFMSs, and many researchers have
studied about transaction management in WFMSs.

Data management is another important role of database technologies in
WFMSs.  As WFMSs must manage many data such as workflow descriptions,
status of progress, activity environments, and activity products, WFMS
products use DBMSs in their backends.  However, the role of DBMSs in WFMSs
is no more than as repositories.  There is no standard data model for
workflows even in the research level.

In this thesis we propose a flexible framework of workflow management
suitable for database technologies, {\em workflow base}.  In this model,
\begin{itemize}
 \item A workflow is defined as a set of objects (\emph{activity
       objects}), each of which corresponds with the unit of work in
       workflows.  Two kinds of flows, horizontal flows and vertical
       flows, are defined.  Both flows are treated as constraints among
       activity objects, hence they are created dynamically from the
       definitions of activity objects.  This makes database management
       of workflows to be easier than ordinary workflow models.
 \item Integrity constraints over workflows are defined on a set of
       activity objects in database.  They can be checked in a similar
       way with ordinary integrity check on database management
       systems.
 \item The concept of workflow instantiation is also defined based on
       generalization/specialization hierarchies of workflows.  This makes 
       relationships among workflows clearer, and workflows more
       reusable.
 \item Execution model of workflow base is defined based on production
       systems.  This model deals with dynamic dispatch of subworks as
       well as ordinary static flows in the same manner.
\end{itemize}

Database features of workflow base are discussed from the various
viewpoints.  Loopback flows are defined using ECA rules, a basic concept
of active databases; extensions on workflow base dealing with time
constraints and resource constraints are introduced; database operations
over workflows based on relational algebra are also introduced, which
realize general purpose view functions and query functions in workflow
management systems; agents as an executer of the units of work are
defined formally as a problem solver in a heterogeneous distributed
environment.  Finally we give a system architecture of workflow base.
