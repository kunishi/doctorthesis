%#!latex main

\chapter{Conclusions}
\label{chap:conclusion}

In this thesis we proposed a flexible framework of workflow management
suitable for database technologies, {\em workflow base}.  

In this model, a workflow is defined as a set of objects (\emph{activity
objects}), each of which corresponds with the units of work in
workflows.  Two kinds of flows in workflow base, horizontal flows and
vertical flows, are treated as constraints among activity objects, hence
they are created dynamically from the definitions of activity objects.
We provide important features of workflow management systems as
applications of database technologies: workflow instantiation based on
generalization/specialization hierarchy of workflows; execution model
based on production systems; loopback flows are defined using ECA rules;
view functions and query functions as a set of database operations;
various constraints on workflows as integrity constraints; agent as a
problem solver under heterogeneous distributed environment.

Workflow base has an important contribution into workflow researches
other than these original features.

When a number of agents cooperate each other, they share some data in
almost all cases.  In this sense, data sharing is essential activity in
cooperative work, as noted in \cite{greif:toois87}, and database will
surely be an infrastructure of groupware technology.
However, it is not in today's groupware.  Why?

I think one of the reasons is a gap between groupware from database
point of view and database from groupware point of view.  From database
point of view, groupware is software on a distributed environment,
sometimes executed automatically; from groupware point of view, on the
other hand, database is software to make capabilities of human beings to
be higher.  Both viewpoints are true; database technologies supporting
cooperative work should satisfy the requirements from both viewpoints.

Workflow base supports well-structured cooperative work from both
viewpoints: to provide a formal but simple workflow model with concrete
execution semantics from database point of view; to provide general and
powerful query language on workflows from groupware point of view.

In this sense, research of query languages on workflow base is the most
important future work.  Though an operation set on workflow base
proposed in Chapter \ref{chap:dbop} is based on relational algebra, it
is quite exhaustive.  From the algebraic point of view, relational
algebra has many good properties: equality with relational calculus,
relationships with Datalog, etc.  Refinement of the operation set to
establish ``workflow algebra'' would give powerful and easy-to-use
workflow management systems, as relational database did in the database
area.
