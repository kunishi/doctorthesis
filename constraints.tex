%#!latex main

\chapter{Extensions on Workflow Base}
\label{chap:ext}

%\section{Motivations}
%\label{sec:ext:intro}

In Chapter \ref{chap:wfbase}, a workflow is modeled as a set of activity
objects, each of which consists from its inputs and outputs, the
responsible agents, and subworks.  This simple modeling is useful to
give formal and powerful frameworks into WFMSs.

On the other hand, real office works are more complex than this modeling.
First they might have feedback loops in most cases; that is, when the
results of some works in workflows do not satisfied the previously defined
requirements, the works or a sequence of the works will be redone until
the requirements are satisfied.  However, the workflow model in Chapter
\ref{chap:wfbase} cannot support feedback flows.

Second weakness of the workflow model in Chapter \ref{chap:wfbase} is
about the constraints of office works.  There are several constraints
about the resources of the organizations, such as deadlines, funds,
materials, persons, etc.  These constraints affect the workflow
structures and the instantiation process of workflows.  These affections
are occurred dynamically even in the execution phase of workflows.
Moreover, the constraints interfere each other.  For example, the agent
responsible with some work is also responsible with its subworks whose
responsible agents are not defined.

In this Chapter, we discuss some extensions to workflow base to deal
with more complex office works.  First we show the way to realize
loopback flows into workflow base, using ECA rules, in Section
\ref{sec:loopback}.  In Section \ref{sec:timeConstraints} and Section
\ref{sec:resourceConstraints} extensions for dealing with \emph{time
constraints} and \emph{resource constraints} are discussed,
respectively.  In Section \ref{sec:timeConstraints}, after introducing
\emph{time plugin} for extending activity objects with time attributes,
we discuss interferences between time constraints and flows in workflow
base, scheduling problem on workflow base, and time adjustment.  In
Section \ref{sec:resourceConstraints}, two resource constraints,
resource sum equality and resource sum inequality are introduced, and
the way for resource reallocation which is done when the resource
constraints are violated is discussed.  And finally, we show in Section
\ref{sec:flowbylock} dynamic horizontal flows caused by exclusive lock
mechanism.

\section{Loopback Flows}
\label{sec:loopback}

In order to deal with feedback flows of workflows, we must provide
conditional flow control mechanisms.  Some conditions for occurring
feedback are evaluated at the source node of feedback flows, and
feedback is occurred when and only when the conditions are satisfied.
If the conditions are not satisfied, feedback is not occurred and
ordinary flows are activated.  In this sense, flow branch mechanism should be
provided at the source node.  

%At the sink node of loopback flows, on the other hand, 

We use {\em Event-Condition-Action (ECA) rules} \cite{mccarthy:sigmod89}
to realize feedback flows in WFB\@.  ECA rule is a basic concept of {\em
active database systems}, DBMSs that allows users to specify actions to
be taken automatically when certain conditions arise.  An ECA rule has
three attributes: {\em event}, {\em condition}, and {\em action}.
Intuitively, semantics of the rule is simple: when the event occurs,
evaluate the condition; and if the condition is satisfied, execute the
action.

For two activity objects $a_1, a_2 \in W. a_1
\stackrel{+}{\Longrightarrow} a_2$ where $W$ is a WFT, a {\em feedback
horizontal flow} from $a_2$ to $a_1$ is defined as a special horizontal
flow such as:
\[
 E, a_2 \Longrightarrow a_1
\]
where $E$ is a set of \emph{feedback events}.  Intuitively, this flow
means that the outputs of $a_2$ are sent to $a_1$ when and only when all 
the feedback events in $E$ is occurred.  
Similarly, for activity objects $a_1, \cdots
a_n, a \in W. \{a_1, \cdots, a_n\} \stackrel{+}{\Longrightarrow} a$, a
feedback horizontal flow from $a$ to $\{a_1, \cdots, a_n\}$ is defined as:
\[
 E, a \Longrightarrow \{a_1, \cdots, a_n\}.
\]
Note that we do not deal with feedback vertical flows in this thesis.
From the definitions in Chapter \ref{chap:wfbase}, vertical flows are
defined from the inclusion relationships among activity objects.  Hence
feedback vertical flows, i.e. loop of the partial orderings defined on
inclusion relationships, are beyond from the conventional set theory.

The semantics of feedback horizontal flow is defined using ECA rules:
\begin{center}
$\begin{array}{rl}
 \mbox{If}\ E, a_2 \Longrightarrow a_1, & \mbox{then}\ a_1 
  \Leftarrow E(a_2).\\
 \mbox{If}\ E, a \Longrightarrow \{a_1, \cdots, a_n\}, & 
  \mbox{then}\ a_1 \Leftarrow E(a), \cdots, a_n \Leftarrow E(a).\\
\end{array}$
\end{center}
The left hand side of $\Leftarrow$ is an action; while the right hand
side of $\Leftarrow$ includes a set of events as well as a set of
conditions.  $E(a)$ means a set of events, with message objects passed
into the sink activity object of the feedback horizontal flow.  When the
event set $E$ is occurred, the rules that include $E$ in its right hand
side are activated, then the activity object in its left hand side is
activated.  Remark that any production rules without event part is not
activated in this case.  This means that any ordinary horizontal flows
are activated when a feedback horizontal flows is activated, and vice
versa.

The rules corresponding with feedback horizontal flows are stored in
P-box of the WFT, as same as ordinary horizontal flows.  No extension is 
needed into the semantics of the production system of P-box.

In order to implement facilities to evaluate some conditions when $E$ is
occurred, it is very simple: just put conditions $c_1, \cdots, c_m$ into 
the left hand side of a feedback horizontal flows, as follows:
\[
 E, c_1, \cdots, c_m, a_2 \Longrightarrow a_1.
\]
This flow defines the ECA rule
\[
 a_1 \Leftarrow E(a_2), c_1, \cdots, c_n.
\]
Intuitively the semantics of this rule corresponds to that of the
ordinary ECA rules: When the events $E$ are occurred, the conditions
$c_1, \cdots, c_n$ are evaluated; all conditions are evaluated as true,
then the activity object $a_1$ is activated.  This extension enables
multiple branch of flows, as follows:
\[
 \begin{array}{l}
  E, c_1, a_2 \Longrightarrow a\\
  E, c_2, a_2 \Longrightarrow a'\\
 \end{array}
\]
The first flow occurs when $c_1$ is true, while the second flow occurs
when $c_2$ is true.  That is, flows will branch by the evaluation
results of the conditions $c_1$ and $c_2$.

\section{Time Constraints}
\label{sec:timeConstraints}

In general, several constraints about time can be considered on
workflows such as:
\begin{itemize}
\setlength{\itemsep}{0mm}
\setlength{\parskip}{0mm}
 \item \emph{startline constraints}.  When the process can be started
     from.
 \item \emph{deadline constraints}.  When the process must be finished
     by.
 \item \emph{duration constraints}.  Constraints about time duration
     among processes.
 \item \emph{prediction constraints}.  Constraints about estimation time 
     of the process.
\end{itemize}

We show some examples of each constraint:
\begin{itemize}
 \item startline constraints.

       Consider again the workflow example in Figure \ref{f:review}.
       Review report format is distributed to
       the reviewers by ftp, but the preparation of ftp service was not
       in time for dispatch of submitted papers.  In such a case, the
       reviewers must wait for writing review reports until ftp service
       is available.

       \begin{sloppypar}
	In this example, startline constraint plays an role of implicit
       	flows.  The workflow in Figure \ref{f:review} does not explicitly
       	describe review report sheets as message objects through vertical
       	flows from ``dispatch-1'' to ``dispatch-2'', or from
       ``dispatch-2'' to ``review''.
       \end{sloppypar}
 \item deadline constraints.

       In Figure \ref{f:review}, as deadline of the process ``send'' is
       fixed at the stage of call-for-papers, each unit of work in the
       workflow has a deadline determining backwardly from the deadline
       of ``send''.
 \item duration constraints.

       ``Receive'' process in Figure \ref{f:review} must be started at
       least one day after, for example, from receiving a submitted
       paper.

       In some cases, processes in workflow must be executed
       simultaneously.  For example, ``send-review-letters'' process in
       Figure \ref{f:review} must be started at the same time for all
       authors of the submitted papers.  This kind of simultaneity can
       be considered as a duration constraint of no duration.
 \item prediction constraints.

       When a reviewer receives submitted papers, he estimates the time
       to be required for the reviews, and then decides whether to
       undertake reviewing.  If he decides to undertake reviewing but he
       is too busy to review all papers by himself, he may redistribute
       the papers to those who are working under him.  In such a case,
       he estimates how long each staff takes to review a paper.
\end{itemize}

In order to deal with these constraints, we propose \emph{time
constraints} on workflow-base.  

\subsection{Time Plugin}
\label{sec:timeplugin}
First, \emph{time plugin} for an activity object $plug_T$ is defined 
as follows:
\[
 plug_T \mydefi (s, e, d, p)
\]
where $s$ is the time $a$ must be started; $e$ is the time $a$ must be
finished by; $d$ is the duration period which $a$ must be done; $p$ is
the prediction period of $a$.  
An activity object can be \emph{plugged in} with time plugin $plug_T$
such as:
\[
 a_T =(I, O, P, S, s, e, d, p)
\]
where $a = (I, O, P, S)$ is an activity object.  A WFT consisting from
activity objects with time plugin is called a \emph{WFT with time
plugin}.

To execute $a_T$ successfully, $a_T$ must satisfy these two inequalities:
\begin{enumerate}
 \item $e-s \geq d$ (\emph{duration inequality}).  Because $a_T$ must
       be executed in the time period $d$, $a_T$ cannot satisfy either
       $s$ or $e$ if $a_T$ does not satisfy this inequality.
 \item $d \geq p$ (\emph{estimation inequality}).  As $a_T$ must be
       executed in the time period $d$, the estimate time of $a_T$ must
       be less than $d$.
\end{enumerate}

To apply WFTs with time plugin to real works, they are instantiated in
the similar way with instantiation of WFTs without time plugin in
Section \ref{sec:workflowinstance}.  Consider a domain $\mathcal{T}$ of
time and a domain $\mathcal{D}$ of time period, with partial orderings
$\sqsubseteq_\mathcal{T}, \sqsubseteq_\mathcal{D}$, respectively.  For
an activity object with time plugin $a_T = (I, O, P, S, s, e, d, p)$, we
consider assignment $\theta$ of $s$ to $s'$, $e$ to $e'$, $d$ to $d'$,
and $p$ to $p'$, where  $s' \sqsubseteq_\mathcal{T} s, e'
\sqsubseteq_\mathcal{T} e, d' \sqsubseteq_\mathcal{D} d, p'
\sqsubseteq_\mathcal{D} p$, respectively, in addition to the ordinary
assignment into activity objects.  Instantiation of $a_T$ is
defined by an assignment $\theta$ to $a_T$, denoted as $a_T\theta$.  An
assignment $\theta$ to a WFT with time plugin $W_T$ is defined as
$W_T\theta = \{a_{T1}\theta, \cdots, a_{Tn}\theta\}$.  

If no assignment of $s$ is included in $\theta$, it is assumed that an
assignment $s/c$ is omitted, where $c$ is the current time on the agent
$P'$.

An assignment $\theta$ including either $e/e'$ or $d/d'$ is called
a \emph{correct assignment on time plugin}.  In a correct assignment on
time plugin, 
\begin{itemize}
 \item it is assumed that an assignment $e/s'+d'$ is omitted if $e/e'$
       is not included;
 \item it is assumed that an assignment $d/e'-s'$ is omitted if $d/d'$
       is not included.
\end{itemize}
If $p/p'$ is not included in a correct assignment on time plugin, it is
assumed that $p/d'$ is omitted.

\subsection{Time Constraints over Flows}
\label{sec:tcFlows}

Time constraints of activity objects which connect by horizontal or
vertical flows interfere each other.  In this section, relationships between
time constraints and the flows in workflow base are discussed.

First consider horizontal flows.  For activity objects with time plugin
$a_1$ and $a_2$ that have a horizontal flow $a_1 \Longrightarrow a_2$,
these inequalities must be satisfied:
\begin{enumerate}
\setlength{\itemsep}{0mm}
\setlength{\parskip}{0mm}
 \item $s_1 + p_1 \leq s_2$
 \item $e_1 + p_2 \leq e_2$
 \item $e_1 \leq s_2$
\end{enumerate}
Similarly, for a horizontal flow $\{a_1, \cdots, a_n\} \Longrightarrow
a$, these inequalities must be satisfied:
\begin{enumerate}
\setlength{\itemsep}{0mm}
\setlength{\parskip}{0mm}
 \item $\max(s_1 + p_1, \cdots, s_n + p_n) \leq s_2$
 \item $\max(e_1 + p, \cdots, e_n + p) \leq e$
 \item $\max(e_1, \cdots, e_n) \leq s$
\end{enumerate}
These are obvious from the definition of horizontal flows.

For a vertical flow $a \longrightarrow a_i$, these inequalities must be
satisfied:
\begin{enumerate}
\setlength{\itemsep}{0mm}
\setlength{\parskip}{0mm}
 \item $s < s_i$
 \item $e > e_i$
 \item $d > d_i$
 \item $p > p_i$
\end{enumerate}
Note that $p$ may not be larger than $\Sigma_{i=1}^n p_i$ because some
of the sub activity objects may be executed concurrently.

\subsection{Scheduling}
\label{sec:scheduling}

An agent generally has a number of to-do works at the same time.  In
such a case, the agent must do scheduling among the works to put to-do
priorities to them.  In this section, we discuss scheduling problems
using the time constraints defined in the previous section.

To simplify the discussions, we first consider scheduling between two
activity objects.  Let
\begin{eqnarray*}
 a_1 & = & (I_1, O_1, P, \emptyset, s_1, e_1, d_1, p_1)\\
 a_2 & = & (I_2, O_2, P, \emptyset, s_2, e_2, d_2, p_2)
\end{eqnarray*}
be activity objects with time plugin.  These two activity objects are
both responsible by an agent $P$ because both have empty set of sub
activity objects.  From the time constraints point of view, we can
categorize relationships between $a_1$ and $a_2$ into two cases: $(s_1 \leq
s_2) \wedge (e_1 \leq e_2)$ and $(s_1 \leq s_2) \wedge (e_2 \leq e_1)$.
Scheduling of $a_1$ and $a_2$ is as follows:
\begin{itemize}
 \item In case of $(s_1 \leq s_2) \wedge (e_1 \leq e_2)$, 
       \[
	\begin{array}{ll}
	 p_1, p_2 & \mbox{if}\ s_1 + p_1 + p_2 \leq e_2 \\
	 \mbox{fail} & \mbox{if}\ s_1 + p_1 +p_2 > e_2 \\
	\end{array}
       \]
 \item In case of $(s_1 < s_2) \wedge (e_2 < e_1)$,
       \[
        \begin{array}{ll}
	 p_{11}, p_2, p_{12} & \mbox{if}\ e_2 \leq s_1 + p_1 + p_2 \leq e_1 \\
	 & \mbox{and if}\ p_1\ \mbox{can be divided into subworks}\\
	 p_1, p_2 & \mbox{if}\ s_1 + p_1 + p+2 \leq e_2\\
	 \mbox{fail} & \mbox{if}\ s_1 + p_1 + p_2 > e_1\\
	\end{array}
       \]
\end{itemize}
``fail'' means that scheduling of these two activity objects is failed.
In this case, the agent $P$ must negotiate with another agents to make
the time constraints of the activity objects more loose.  We do not discuss
the negotiation process in this thesis.

Scheduling problem among more than three activity objects is much more
difficult than that of two activity objects.  In actual, the problem is
treated as an linear programming problem.  The detailed discussions from
this point of view is beyond this thesis.  

\subsection{Time Adjustment}
\label{sec:timeadj}

In the discussions above, we assume that all agents share one global
clock.  However, this assumption is too strict under distributed
environment: each agent in general have the different clock, and there
are time lags among them.  In this section, we extend the time
constraints on workflow base by loosening this assumption.

First we extend time plugin as follows:
\[
 plug_T \mydefi (c, s, e, d, p)
\]
and an activity object with time plugin $a_T$ is extended as:
\[
 a_T \mydefi (I, O, P, S, c, s, e, d, p)
\]
where $c$ is the current time of the clock which $P$ has.

Consider a situation that the agent $P$ dispatches $a_T$ to the
agent $P'$.  In this case, $P'$ adjusts time plugin of $a_T$ as follows: 
\[
 s := s - a - t;\;  e := e - a - t;
\]
where $a = c - c'$; $t$ is the transmission time from $P$ to $P'$.

\section{Resource Constraints}
\label{sec:resourceConstraints}

In this section, we give a workflow of publishing process as another
example of workflow which have more severe constraints about resources
than the reviewing process.

Figure \ref{fig:publishwf} shows an illustrative example of publishing
process of magazine\footnote{In Figure \ref{fig:publishwf}, horizontal
flows and vertical flows are represented as thick arcs and thin arcs,
respectively.}.  The process is defined as a sequence of two units:
producing posters and producing magazine.  Each unit consists from some
subprocesses.  In ``planning'' process, an outline of the printing
matter such as its design, funds, total pages and contents in case of
magazine, etc., is determined.  Then the requests of producing contents
are sent to authors.  In case of producing magazine, all the contents
are gathered for editing.  Finally, camera readies are sent to the
printing house for printing.  We assume that posters and magazines use
the same logo; hence a horizontal flow from ``poster printing'' to
``magazine printing'' is put to this workflow because the logo is first
designed in ``poster printing'' process, then sent to ``magazine
printing'' process.

\begin{figure}
\begin{center}
{\footnotesize
 \setlength{\unitlength}{0.72mm}
 %#!latex main

% Workflow Example for Publishing Process

\begin{picture}(180,100)(0,0)
\put(5,70){\begin{picture}(40,20)(0,0)
	     \put(20,10){\oval(30,16)}
	     \put(20,10){\makebox(0,0){\shortstack{poster\\production}}}
	    \end{picture}}
\put(25,50){\begin{picture}(40,20)
	     \put(20,10){\oval(30,16)}
	     \put(20,10){\makebox(0,0){planning}}
	    \end{picture}}
\put(60,60){\Thicklines\vector(1,0){10}}
\put(65,50){\begin{picture}(40,20)(0,0)
	     \put(20,10){\oval(30,16)}
	     \put(20,10){\makebox(0,0){\shortstack{request\\contents\\production}}}
	    \end{picture}}
\put(100,60){\Thicklines\vector(1,0){10}}
\put(105,50){\begin{picture}(40,20)(0,0)
	     \put(20,10){\oval(30,16)}
	     \put(20,10){\makebox(0,0){printing}}
	    \end{picture}}
\put(5,25){\begin{picture}(40,20)(0,0)
	      \put(20,10){\oval(30,16)}
	      \put(20,10){\makebox(0,0){\shortstack{magazine\\production}}}
	     \end{picture}}
\put(25,5){\begin{picture}(40,20)
	     \put(20,10){\oval(30,16)}
	     \put(20,10){\makebox(0,0){planning}}
	    \end{picture}}
\put(60,15){\Thicklines\vector(1,0){10}}
\put(65,5){\begin{picture}(40,20)(0,0)
	     \put(20,10){\oval(30,16)}
	     \put(20,10){\makebox(0,0){\shortstack{request\\contents\\production}}}
	    \end{picture}}
\put(100,15){\Thicklines\vector(1,0){10}}
\put(105,5){\begin{picture}(40,20)(0,0)
	      \put(20,10){\oval(30,16)}
	      \put(20,10){\makebox(0,0){\shortstack{gather\\contents}}}
	     \end{picture}}
\put(140,15){\Thicklines\vector(1,0){10}}
\put(145,5){\begin{picture}(40,20)(0,0)
	     \put(20,10){\oval(30,16)}
	     \put(20,10){\makebox(0,0){printing}}
	    \end{picture}}
\put(25,72){\Thicklines\vector(0,-1){29}}
\put(33,72){\vector(1,-1){4}}
\put(33,27){\vector(1,-1){4}}
\end{picture}

}
\end{center}
\caption{Workflow Example: Publishing Process}
\label{fig:publishwf}
\end{figure}

There are many constraints about resources in the workflow in Figure
\ref{fig:publishwf}.  For example, the sum of the costs for two printing 
processes is determined beforehand; the number of total pages for the
magazines is also determined in the ``planning'' process.  Agent
determines the costs of the subprocesses, or the number of pages for
each articles, with satisfying the resource constraints.  In the next
section we formalize this kind of resource constraints about its sum.

\subsection{Resource Sum Constraints}
\label{sec:sumconstraint}

Let $amount_R$ allocation amount of a resource $R$.  \emph{Resource
plugin} $plug_R$ on a resource $R$ is defined as
\[
 plug_R \mydefi (amount_R).
\]
That is, $plug_R$ is a tuple with only one attribute.  For an activity
object $a = (I, O, P, S)$, an \emph{activity object with resource
plugin} $a_R$ is defined as 
\[
 a_R \mydefi (I, O, P, S, amount_R).
\]
For a WFT $W=\{a_1, \cdots, a_n\}$, a \emph{WFT with resource plugin}
$W_R$ is defined as $W_R = \{{a_1}_R, \cdots, {a_n}_R\}$.

All resource plugins and time plugin are compatible with each other.
For example, an activity object with time plugin can be plugged in with
a resource plugin, and vice versa.

For a WFT with resource plugin $W_R = (I, O, P, S, amount_R)$, we define 
the \emph{resource sum equality} on $R$ as:
\[
 \forall a \in W_R \wedge S \neq \emptyset.\ 
 amount_R = \sum_{a' \in S} a'.amount_R
\]
Intuitively, this equality means that the sum of the resources for
subprocesses must be equal to the allocation amount of the parent
process.

Note that some of the resources does not satisfy resource sum equality.
Total costs in the workflow of Figure \ref{fig:publishwf} is an example
because some costs must be used in ``poster printing'' process or
``magazine printing'' themselves.  Hence we consider another
resource constraints, called \emph{resource sum inequality}:
\[
 \forall a \in W_R \wedge S \neq \emptyset.\ 
 amount_R \geq \sum_{a' \in S} a'.amount_R
\]

As all resources in subprocesses are what the parent process allocated
for the subprocesses, resource sum inequality must be satisfied in all
resources.  Resource sum equality and resource sum inequality are called 
generally as \emph{resource constraints}.

\subsection{Resource Reallocation}
\label{sec:realloc}

If one of the subprocesses can not satisfy the resource constraints
determined by the agent of the parent process, the resource constraints
of the parent process are also not satisfied.

See again the example in Figure \ref{fig:publishwf}.  Consider a
situation that an author writes his article of less pages than that of
previously determined.  This causes constraint violation about total
pages.  Hence the publisher must resolve the violation by some means: by
requesting the author to write more pages, by requesting another author
to write more pages, or by changing total pages of the magazine.

If an activity object with resource plugin $a_R$ in WFT does not satisfy
resource constraints, the agent of $a_R$ sends a message ``fail
allocation in $R$'' to the agents who are responsible with activity
objects which $a_R$ belongs.  When an agent receives the message ``fail
allocation in $R$'', it tries to reallocate resource $R$ into sub
activity objects.  If the reallocation succeeds, the agent sends the
reallocation results to the sub activity objects.  If it fails, the
agent sends ``fail allocation in $R$'' to parent activity objects again.
We do not discuss about negotiations among agents during the
reallocation process.

\section{Horizontal Flows by Exclusive Locks}
\label{sec:flowbylock}

Under the situation that data is shared among agents in the same
cooperative work, it is important to provide lock mechanisms for keeping
integrities of data.  Lock mechanisms previously proposed can be
classified into two categories: exclusive locks and shared locks.
Exclusive locks permit only one user to access locked data; on the other
hand, shared locks permit a number of users to access locked data while
locking.  In workflow base, data sharing method is not included in its
definitions, hence any lock mechanisms can be used.

When exclusive locks are used for data sharing, time constraints among
activity objects are newly created.  Consider an example that two
activity objects $a_1$ and $a_2$ in the same WFT share a file $f$ using
exclusive lock.  If $a_1$ uses $f$ with exclusive lock, $a_2$ must wait
until the lock is released.  Therefore a time constraint $e_1 < s_2$ is
newly created.

This phenomenon can be treated in a uniform way on workflow base.  When
$a_1$ begins to use $f$ with exclusive lock, a special message object
``release notification'' is added into $O$ of $a_1$ and $I$ of $a_2$.
When $a_1$ releases the lock, ``release notification'' is sent to the
WFT from $a_1$, and $a_2$ catches it as an input.  Note that a
horizontal flow $a_1 \Longrightarrow a_2$ is newly created when
``release notification'' is added.  According to the discussions in
Section \ref{sec:timeConstraints}, a time constraint $e_1 < s_2$ is also 
created automatically.

\section{Summary}
\label{sec:ext:conclu}

In this Chapter, we discussed four extensions to workflow base.  First
we showed the way to realize loopback flows into workflow base, using
ECA rules.  Next we discussed about two kinds of constraints, time
constraints and resource constraints, and about various properties among
them: interferences of these constraints with flows in workflows and
their solutions; applications such as scheduling.  And finally, we showed
that exclusive lock mechanism causes horizontal flows dynamically.

In order to apply workflow base to real cooperative works, some
extensions discussed in this chapter must be needed.  However, workflow
base is so suitable for database technologies that extensions can be
easily done using database technologies.  In other words, these
extensions prove high affinities of workflow base with databases.
