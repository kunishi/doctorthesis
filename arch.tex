%#!latex main

\chapter{System Architecture of Workflow Base}
\label{chap:arch}

In this chapter, we discuss about how to implement workflow base, mainly
from system architecture point of view.  Though workflow base is closely
related to database systems, it has various features not found in
traditional database systems, such as an execution model based on
production systems.  First we show a system interface between workflow
base and application programs.  After that, overall of the architecture
and each functional components are described.

%\section{System Requirements}

%As stated above, workflow model in workflow base is similar with data
%models such as relational data model, object-oriented data model,
%deductive data model, etc., but not the same.  Therefore an mechanism to 
%treat workflow model must be implemented on top of the traditional
%database systems.
%
%In a similar reason, an interpreter for workflow operations must be
%realized as a different software component with database query
%interpreter.
%
%The present status of each WFTs must be managed.  Remark that each
%agents assigning the units of work in WFTs may do a number of work
%concurrently.  Even in this case, the roles of the agents in each WFTs
%must be distinguished in workflow base.

\begin{sloppypar}
 \section{The Interface Between Applications and Workflow Base}
\end{sloppypar}

Workflow base has two roles for its application programs, called {\em
workflow clients}: as a workflow database and as a workflow server.
That is, workflow base is more than a workflow repository in workflow
management systems; it can behave as a server of workflow management
systems itself because it provides an execution model of each
workflows.

Therefore workflow base has two modules in its interface: interface to
the workflow database and interface to the workflow server.  

The former provides database operations described in Chapter
\ref{chap:dbop}.  Workflow clients uses this interface as querying
workflow base, such as retrieving, reorganizing workflows, etc.

Meanwhile the latter provides the protocol between the capsule and the
problem solver mentioned in Chapter \ref{chap:agentReconsidered}.  In
the coordination system point of view, all agents related with a
collaborative work of workflows are implemented in the workflow server
and coordinate each other in the style of Chapter
\ref{chap:agentReconsidered}.  A workflow client is denoted as a problem
solver in this structure.

\section{The System Architecture of Workflow Base}

\begin{sloppypar}
 Workflow base organizes from four functional components: Database, WFT
 organizer, WFT interpreter, and Query processor.  An overview of the
 system architecture is shown in Figure \ref{fig:wfarch}.
\end{sloppypar}

\begin{figure}
\begin{center}
\setlength{\unitlength}{0.4mm}
{\small
 %#! latex main

\begin{picture}(220,300)(0,0)
% Database
\put(80,10){\begin{picture}(60,60)(0,0)
	     \put(30,50){\ellipse{60}{20}}
	     \put(0,10){\line(0,1){40}}
	     \put(60,10){\line(0,1){40}}
	     \put(0,10){\spline(0,0)(5,-6)(30,-10)(55,-6)(60,0)}
	     \put(30,25){\makebox(0,0){DB}}
	    \end{picture}}
% WFT organizer
\put(60,100){\framebox(100,20){WFT Organizer}}
% WFT interpreter
\put(0,160){\framebox(100,20){WFT Interpreter}}
% Query Processor
\put(120,160){\framebox(100,20){Query Processor}}
% Workflow Base
\put(-10,0){\Thicklines\framebox(240,200)[bl]{{\bf Workflow Base}}}
% Client
\put(110,260){\begin{picture}(120,40)(60,20)
	       \put(60,20){\thicklines\oval(120,40)}
	       \put(60,20){\makebox(0,0){{\bf Workflow Client}}}
	      \end{picture}}
%% Arcs
% from Database to WFT organizer
\put(105,100){\vector(0,-1){40}}
\put(115,60){\vector(0,1){40}}
\put(65,80){\makebox(0,0){WFT parts}}
\put(155,80){\makebox(0,0){WFT parts}}
% WFT organizer and WFT interpreter
\put(90,120){\vector(-3,4){30}}
\put(50,140){\makebox(0,0){WFT}}
% WFT organizer and Query processor
\put(130,120){\vector(3,4){30}}
\put(170,160){\vector(-3,-4){30}}
\put(125,145){\makebox(0,0){Results}}
\put(175,135){\makebox(0,0){Query}}
% WFT interpreter and WF Client
\put(40,180){\vector(3,4){45}}
\put(75,240){\vector(-3,-4){45}}
\put(20,215){\makebox(0,0){WF access}}
\put(95,210){\makebox(0,0){Messages}}
% Query processor and WF Client
\put(180,180){\vector(-3,4){45}}
\put(145,240){\vector(3,-4){45}}
\put(130,220){\makebox(0,0){Results}}
\put(195,210){\makebox(0,0){Query}}
\end{picture}

}
\end{center}
\caption{System Architecture of Workflow Base}
\label{fig:wfarch}
\end{figure}

\emph{Database} stores all information about WFTs such as their
definitions, all activity objects, all P-box and C-box components,
assignments, progress status, etc.  Remark that this is no need for
object-oriented database, though we use the term ``objects'' in workflow
base.  Activity object is not encapsulated because it has no concepts of
methods.  Hence it is more similar to tuple in relational database than
object in object-oriented database.

\emph{WFT organizer} constructs WFTs from the information in Database,
and decomposes WFTs in data who belongs to the schema in Database.  In
other words, WFT organizer translates WFTs and schema of Database each
other.  It plays a role of hiding database implementation --- relational
database, object-oriented database, etc. --- from other components of
workflow base.

\emph{WFT interpreter} interprets and executes WFTs based on its
execution model proposed in Section \ref{sec:execmodel}.  It must keep
the current status of all WFTs currently on work.  Hence, in
implementation level, it is newly invoked every time a WFT is
available.  That is, WFTs now on work and WFT interpreters currently
running are related with one-to-one correspondence.  Integrity
constraints on WFTs are checked by WFT interpreter.

\emph{Query processor} is an interpreter of workflow operations shown in 
Chapter \ref{chap:dbop}.  It mainly manages queries on workflow base and 
the operations which changes WFTs or activity objects directly.  On the
other hand, WFT interpreter manages operations on WFTs in the frame of
the execution model of WFT.

Agents in real world --- humans in most cases --- use \emph{workflow
clients} to communicate with workflow base.  In this sense, a
client-server system consisting from WFT clients and WFT interpreters
corresponds with traditional workflow management system.

\section{Summary}

In this chapter system architecture of workflow base is discussed.
Though the roles of each software components are briefly mentioned, more
discussions about the architectures of them should be needed.  In
especial, these two points must be discussed: interpretation procedure
for the execution model WFT interpreter should have, and database schema
for modeling WFTs.
