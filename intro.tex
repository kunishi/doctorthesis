%#!latex main

\chapter{Introduction}
%\pagenumbering{arabic}
%\setcounter{page}{1}

\section{Background}
\label{sec:background}

During the recent two or three years, computer networks permeate into 
several areas of our everyday life.

The Internet started just as a computer wide area network among the 
computer research sections of universities, companies, and some public 
organizations.  At that time, the users of the Internet communicate 
each other through the Internet by electronic mails, network news, or 
character-based realtime applications such as electronic phones.  
These are only the usages of the Internet.

\begin{sloppypar}
 However, this situation around the Internet has been drastically 
 changing as several technologies are progressed in the areas of 
 computer and digital networks.  Now the Internet is no longer an 
 experimental object for computer researchers; it is an infrastructure 
 of several business activities, including personal activities.  
 Applications on the Internet has also changed from character based to 
 multimedia based.  World Wide Web (WWW), a multimedia hypertext over 
 the Internet, is an example of hot multimedia based applications 
 among the Internet users.
\end{sloppypar}

Groupware is another hot application over the Internet.  It is a
general term for the technologies supporting human communications or
cooperative work over the computer networks, such as electric meetings,
cooperative writing, etc.  Ellis \cite{ellis:cacm91} defined the term
``groupware'' as:
\begin{quote}
 Computer-based systems that support groups of people engaged in a
 common task (or goal) and that provide an interface to a shared
 environment.
\end{quote}
Although the concept of groupware is proposed in 1978
\cite{johnson-lentz82:groupware}, it had not been practical until the
recent years because of the lack of its infrastructures, such as
computer powers and the network bandwidth.  As this problem has been
gradually solved, many researches and developments of groupware are
carried out.

Ellis categorizes groupware into four types from notions of time and space
\cite{ellis:cacm91}.  They are:
\begin{itemize}
 \setlength{\itemsep}{0mm}
 \setlength{\parskip}{0mm}
 \item  \emph{Face-to face interaction}.  This type supports cooperation
	in the same place and the same time.  Electric meeting room
        system \cite{stefik:cacm87, stefik:toois87} can be shown as its
	example.
 \item  \emph{Asynchronous interaction}.  This type supports cooperation
	in the same place but in the different times.
 \item  \begin{sloppypar}
	 \emph{Synchronous distributed interaction}.  This type supports
	 cooperation in the same time but in the different places.
	 Group editor \cite{foster:cscw86} and distributed
	 electric meeting system \cite{crowley:cscw90} are some of the
	 examples.
	\end{sloppypar}
 \item  \emph{Asynchronous distributed interaction}.  This type supports
	cooperation in the different times and the different place.
	These are the examples: task coordination systems
        \cite{winograd:book86, flores:toois88}, information filtering systems
	\cite{malone:toois87}, office procedure systems
	\cite{ishii:jip91, suchman:toois83, croft:toois84}, and
	hypertexts \cite{conklin:toois88}.
\end{itemize}

\begin{sloppypar}
 Workflow management system (WFMS) \cite{georgakopoulos:jdps95} is one
 of the remarkable groupware not only in practical business area but also
 in research area.  WFMSs, belonging into the groupware supporting
 asynchronous distributed interaction according to Ellis's
 categorization, mainly support structured collaborative works using
 explicitly defined flows of works (workflows).  They controls the
 invocation order in the workflows and manages several resources of the
 work automatically.
\end{sloppypar}

As a number of workflows run concurrently using shared resources of
organizations, transaction management with concurrency control is an
important technology for WFMSs.  In this sense database technologies are
indispensable for the infrastructure of WFMSs, and many researchers have
studied about transaction management in WFMSs
\cite{georgakopoulos:ijicis94, rusinkiewicz:adbis94,%
krishnakumar:jdps95, alonso:icde96}.

Another contribution of database technologies for WFMSs is data
management.  As WFMSs must manage many data such as workflow
descriptions, status of progress, activity environments, and activity
products, WFMS products use DBMSs in their backends.  However the role of
DBMSs in WFMSs is no more than as repositories.  There is no standard
data model for workflows even in the research level, though some
researchers pointed out the importance of data sharing in computer
supported cooperative work \cite{greif:toois87} and workflow data models
\cite{alonso:nsf96}.

If DBMSs support WFMSs more closely in their data management, WFMSs can
provide more useful and powerful functions.  This is our
standpoint\cite{kunishima:ipsjdbs95-07, kunishima:codas96,%
yokota:upmail96, yokota:lnai97}.  DBMSs' supports bring the following
advantages into WFMSs:
\begin{itemize}
 \setlength{\itemsep}{0mm}
 \setlength{\parskip}{0mm}
 \item By managing all workflow descriptions in one DBMS, it is easy to
     resolve duplications or conflicts among the workflows.  This leads
     to efficient workflow management.
 \item Management of workflow hierarchies makes reuse of workflows 
     possible.
 \item Powerful view functions can be provided.  For example, private
     schedule can be obtained as a view of workflows.
 \item Workflows can be easily updated, changed, or reorganized even if 
     they are in progress.
 \item An integrated work environment can be provided by managing both
     product data and process data.
\end{itemize}
As real offices are open \cite{hewitt:toois86}, groupware should support
office works flexibly, even in the procedural works.  Hence these advantages
are useful for flexible workflow management systems.

In this thesis we propose a flexible framework of workflow management
suitable for database technologies, {\em workflow base}.  The features
of this model and the merits are as follows:
\begin{enumerate}
\setlength{\itemsep}{0mm}
\setlength{\parskip}{0mm}
 \item A workflow is defined as a set of objects (\emph{activity
       objects}), each of which corresponds with the unit of work in
       workflows.  This makes database management of workflows to be
       easier than ordinary workflow models.
 \item Two kinds of flows, horizontal flows and vertical flows, are
       defined.  Ordinary workflows can be described by using these
       flows.  Both flows are treated as constraints among activity
       objects, hence they are derived dynamically from the definitions
       of activity objects.
 \item The concept of generalization/specialization workflow hierarchies is
       introduced.  This makes relationships among workflows clearer,
       and workflows more reusable.  The concept of workflow
       instantiation is also defined based on these hierarchies.
 \item A rule-based execution model of workflow is defined.  This model
       deals with dynamic dispatch of subworks as well as ordinary
       static flows in the same manner.
 \item Integrity constraints over workflows are defined.  As they are
       defined on a set of activity objects in database, they can be
       checked in a similar way with ordinary integrity check on
       database management systems.
\end{enumerate}

We also discuss several database features of workflow base from the various
viewpoints.  Loopback flows are defined using ECA rules, a basic concept 
of active databases; extensions on workflow base to deal with time constraints
and resource constraints are defined; database operations over workflows
based on relational algebra are introduced, which realize general purpose
view functions and query functions in workflow management systems;
agents as an executer of the units of work are defined formally as a
problem solver in a heterogeneous distributed environment.  And finally, 
we discuss about system architecture of workflow base.

\section{Outline of the Thesis}
\label{sec:outline}

The remainder of this thesis is as follows.  In Chapter
\ref{chap:preliminaries}, preliminaries for the discussions of the
latter chapters are provided.  First we explain the basic concepts of
workflow management systems and the requirements to workflow management
systems are discussed.  Secondly, brief explanations about definite
clauses, logic programming, production rules, and ECA
(Event-Condition-Action) rules are shown.  We utilize these concepts in
the definition of workflow base and its extensions.  Finally, we explain
the concept of {\em agent} as a heterogeneous distributed cooperative
problem solver.  The term ``agent'' is used in several contexts of
distributed cooperative environment.  We use ``agent'' as an executer of 
the units of work in workflows.  The concept of agent is defined from
this context.  This definition is utilized in Chapter
\ref{chap:agentReconsidered}.

\emph{Workflow base}, a formal model of workflow database, is proposed
in Chapter \ref{chap:wfbase}.  This is the core model of this thesis.
Its basic idea is that a workflow is represented as a set of units of
work and the constraints among the units.  This idea makes workflow
management using database technologies to be easier than other workflow
models.  Section \ref{sec:wfmodel} defines the structure of workflows in
workflow base.  {\em Activity objects} representing the units of work in
workflows, two kind of flows based on message passing between activity
objects and on part-of hierarchy over activity objects, and several
constraints on {\em workflow templates} are defined in this section.
Section \ref{sec:execmodel} gives an execution model of workflow
templates based on production system.  In Section
\ref{sec:workflowinstance} instantiation concept of workflow templates
is defined using specialization hierarchies of workflows.  Based on the
preceding discussions, workflow base is defined in Section
\ref{sec:workflowbase}.

Some extensions on workflow base are discussed in Chapter
\ref{chap:ext}.  Although workflow base defined in Chapter
\ref{chap:wfbase} supports basic functions indispensable for workflows,
some extended features such as loopbacks, time constraints, resources
constraints, etc., are necessary for workflows.  We discuss these
extensions: loopback flows based on ECA rules are defined in Section
\ref{sec:loopback}; time constraints and some applications on them such
as scheduling are shown in Section \ref{sec:timeConstraints}; the
constraints of resources on workflows are discussed in Section
\ref{sec:resourceConstraints}.  We also discuss a method for resource
reallocation in the time of violations in resource constraints in this
section.  And finally, we show that exclusive lock mechanism causes
horizontal flows dynamically in Section \ref{sec:flowbylock}.

Chapter \ref{chap:dbop} describes about database operations on workflow
base.  If the operations on workflows are provided, workflow management
will be more flexible and powerful: View functions on workflows,
dynamic change on workflows with keeping integrity constraints, etc.
As workflow base manages workflows in a style suitable for database
management, the operations on workflows can be easily provided.  In this 
chapter, an operation set based on relational algebra is proposed.

Chapter \ref{chap:agentReconsidered} is devoted into reconsiderations about
agents.  In the previous chapters, we consider agents as an executer of
the units of work.  We give another definition of agents in workflow
base, as a problem solver enclosed in a capsule.  As workflow base is
organized in heterogeneous distributed environment, agents are
essentially also heterogeneous.  Hence in workflow base, mechanisms that
heterogeneous agents must coordinate each other.  We show such a
mechanism by providing an {\em environment} for message-passing between
agents.

We discuss about how to implement workflow base, mainly from system
architecture point of view in Chapter \ref{chap:arch}.  Though workflow
base is closely related to database systems, it has various features not
found in traditional database systems, such as an execution model based
on production systems.  We first investigate system requirements to
implement workflow base, and then show a system architecture of workflow
base.

Related researches are shown in Chapter \ref{chap:relatedworks}, with
comparisons to workflow base.  There are many researches about workflow
management systems.  Moreover, there are also similar concepts as
workflows in various research areas, such as groupware, process
modeling, database, and software process engineering.  In this chapter,
we pick up several related researches from three research areas ---
groupware, process modeling, and database --- and compare them with
workflow base especially from the workflow model point of view.

We conclude with discussions of future work in Chapter
\ref{chap:conclusion}.
