%#!latex main

\newcommand{\ancestor}{\mathop{\rm ancestor}\nolimits}
\newcommand{\descendant}{\mathop{\rm descendant}\nolimits}
\newcommand{\wft}{\mathop{\rm wft}\nolimits}
\newcommand{\wfi}{\mathop{\rm wfi}\nolimits}
\newcommand{\wftgeneral}{\mathop{\rm general}\nolimits}
\newcommand{\wftspecial}{\mathop{\rm special}\nolimits}
\newcommand{\evalexp}{\mathop{\rm eval}\nolimits}

\newcommand{\selectao}{\mathop{\sigma}}
\newcommand{\selectwft}{\mathop{\sigma'}}
\newcommand{\getworkflow}{\mathop{\eta}}
\newcommand{\wfbhide}{\mathop{\psi}}
\newcommand{\wfbshow}{\mathop{\Psi}}
\newcommand{\wfbgeneral}{\mathop{\vartheta}}
\newcommand{\wfbspecial}{\mathop{\Theta}}
\newcommand{\wfbancestor}{\mathop{\vartheta^+}}
\newcommand{\wfbdescendant}{\mathop{\Theta^+}}
\newcommand{\wfbgrouping}{\mathop{\omega}}
\newcommand{\wfbflatten}{\mathop{\Omega}}
\newcommand{\wfbsplit}{\mathop{\gamma}}
\newcommand{\wfbconcat}{\mathop{\Gamma}}
\newcommand{\wfbplugin}{\mathop{\Join^+}}
\newcommand{\wfbplugout}{\mathop{\pi^+}}

\chapter{Database Operations on Workflow Base}
\label{chap:dbop}

As mentioned in Section \ref{sec:pre:wfms}, conventional workflow
management systems have no standard data manipulation language.
This causes several problems into WFMSs: the lack of interoperability
among WFMSs, as pointed out in \ref{sec:pre:wfms}; poor functions for
reorganizing workflows such as workflow reuse, view functions
including the creation of to-do lists, workflow optimization in business 
process reengineering, etc.

Data manipulation language is what database technologies make the
greatest contribution to workflow management, because data manipulation
of general purpose is one of the most essential facilities in DBMSs, and 
therefore many technologies such as SQL are developed.

Workflow base provides a formal model of workflows, which is suitable
for data manipulation of database technologies.  In this chapter, we
give several operations manipulating workflows on workflow base.

They are defined as an extension of ordinary relational algebra.  That
is, these operations get a set of WFTs or a set of activity objects from
WFTs in WFB\@.  The operations are classified into three categories:
\begin{itemize}
 \item Query operations.\\
       A {\it new} set of WFTs (or of activity objects) is obtained from
       the previously defined workflows  No effects over the original
       WFTs (or activity objects).
 \item Operations handling WFTs.\\
       Though some effects may cause over other WFTs, they keep WFTs as
       workflows to be consistent.
 \item Operations handling activity objects directly.\\
       Any effects may cause over WFTs.  Sometimes the effects make WFTs 
       no more than workflows.
\end{itemize}
As some of them do not always keep the closure property of relational
algebra, an operation for keeping closure property is also provided.
The combination of this operation with others makes the closure property
to be kept.

When using these operations, users can organize new workflows from the
workflows in WFB\@.  For example, the following functions are available
using the operations:
\begin{itemize}
\itemsep0mm\parskip0mm
 \item Views over workflows, such as to-do lists.
 \item Investigation of progress.
 \item Workflow reorganization.
 \item Reuse of workflows.
\end{itemize}

Note that any ordinary set operators or operations in relational algebra
can be used for manipulating WFTs or activity objects.  And also note
that the components of an activity object are accessible using dot
notations.  That is, $a.I$ means $I$ of $a$.  Hence the notations such
as $a.I := a.I \cup \{kunishima\}$ can be available.  We do not discuss
these points further in this thesis.

\section{Query Operations}
\label{sec:retrieveop}

\subsection{Selection}

Two selection operations on a WFB can be considered: selection for a set 
of activity objects, and selection for a set of WFTs.

Selection of activity objects, $\selectao_{expr} W$, gets a set of activity
objects satisfying an AO-expression $expr$ from a specified WFT $W$.
AO-expressions are recursively defined as follows:
\begin{itemize}
 \itemsep0mm\parskip0mm
 \item $p \in P$ is an AO-expression, where $p \in {\cal P}$.
 \item $m \in I$ is an AO-expression, where $m \in {\cal M}$.
 \item $m \in O$ is an AO-expression, where $m \in {\cal M}$.
 \item $\ancestor(a)$ is an AO-expression, where $a \in W$.
 \item $\descendant(a)$ is an AO-expression, where $a \in W$.
 \item If both $exp1$ and $exp2$ are AO-expressions, $exp1 \charop
      exp2$ is also an AO-expression, where $\charop$ is a conventional logical
      operator.
\end{itemize}
If $W$ is omitted, a WFT of all activity objects in a WFB, $W_{all}$, is
assumed as $W$.  This assumption is all applicable on other operations.
Evaluation value of an AO-expression $expr$, $\evalexp(expr)$ is defined as
follows.  For an activity object $a' = (I, O, P, S) \in W$,
\[
 \begin{array}{l}
 \evalexp(p \in P) = {\tt true} \chariff p \in P.\\
 \evalexp(m \in I) = {\tt true} \chariff m \in I.\\
 \evalexp(m \in O) = {\tt true} \chariff m \in O.\\
 \evalexp(\ancestor(a)) = {\tt true} \chariff a \longrightarrow a', a' \in W.\\
 \evalexp(\descendant(a)) = {\tt true} \chariff a' \longrightarrow a, a' \in W.\\
 \evalexp(exp1 \charop exp2) = \evalexp(exp1) \charop \evalexp(exp2)
 \end{array}
\]

Selection of WFTs, $\selectwft_{expr} W$, on the other hand, gets a set of
WFTs satisfying a WFT-expression $expr$ from a specified WFT $W$.
WFT-expressions are recursively defined as follows:
\begin{itemize}
 \itemsep0mm\parskip0mm
 \item $\wft(W')$ is a WFT-expression, where $W'$ is a WFT.
 \item $\wfi(W')$ is a WFT-expression, where $W'$ is a WFT.
 \item $\wftgeneral(W')$ is a WFT-expression, where $W'$ is a WFT.
 \item $\wftspecial(W')$ is a WFT-expression, where $W'$ is a WFT.
 \item If both $exp1$ and $exp2$ are WFT-expressions, $exp1 \charop
      exp2$ is also a WFT-expression, where $\charop$ is a conventional logical
      operator.
\end{itemize}
Evaluation value of a WFT-expression $expr$, $\evalexp(expr)$ is
defined as follows.  For a WFT $W''$,
\[
 \begin{array}{l}
  \evalexp(\wft(W')) = {\tt true} \chariff W' \sqsubseteq W'',
      \neg\exists W_1. W' \sqsubseteq W_1 \sqsubseteq W''.\\
  \evalexp(\wfi(W')) = {\tt true} \chariff W'' \sqsubseteq W',
      \neg\exists W_1. W'' \sqsubseteq W_1 \sqsubseteq W'.\\
  \evalexp(\wftgeneral(W')) = {\tt true}  \chariff W' \sqsubseteq W''.\\
  \evalexp(\wftspecial(W')) = {\tt true} \chariff W'' \sqsubseteq W'.\\
  \evalexp(exp1 \charop exp2) = \evalexp(exp1) \charop \evalexp(exp2)
 \end{array}
\]

Both selection operations are similar with selection operator in
relational algebra.  However, there is a difference between those two:
the result of $\selectao$ is not always a closed WFT, meanwhile the result
of $\selectwft$ is always closed.  That is, when $\selectao$ is applied on a
workflow, the result is not always a workflow.

Now we provide an operation to get a workflow from a set of activity
objects, {\em get workflow} operation, noted as $\getworkflow$.  Its
syntax is $\getworkflow_W a$ or $\getworkflow_W \{a_1, \cdots, a_n\})$.
It gets a maximal workflow including an activity object $a$ or activity
objects $\{a_1, \cdots, a_n\}$ respectively, from a specified WFT $W$.

By using $\getworkflow$, a workflow can be obtained from the result of
$\selectao$.

\subsection{Operations on Has-a Hierarchies}

In general, a WFT organizes hierarchically by vertical flows.  Hence
there exists a requirement to hide some sub activity objects from a WFT to
look at an overview of work.  In order to satisfy the requirement, we
provide operations {\em hide} and {\em show} as the reverse operator of
{\em hide}.

Hide operation, noted as $\wfbhide(W_1, W_2)$, returns a WFT $W_1 - W_2$
where $W_1$ is a WFT, $W_2$ is a closed WFT, and $W_2 \subseteq W_1$.
Intuitively, this operator hides all the descendants of the activity
objects in $W_2$.  In other words, $\wfbhide$ works as an abstraction
operation on a WFT.  The obtained WFT does not always satisfy closed
property.

Show operation, denoted as $\wfbshow W_1$, is the reverse operation of
$\wfbhide$.  For a WFT $W_1$, this operation returns a WFT
\[
 W_1 \cup \bigcup_{a \in W_1} \descendant(a).
\]
$W_1$ may not be closed.  On the other hand, the obtained WFT is
always closed.
Intuitively this operation shows all the descendants of $W_1$.  In other 
words, $\wfbshow$ works as a detailing operation on a WFT.

\subsection{Operations on Specialization Hierarchies}

Specialization hierarchies of WFTs are useful for retrieving WFTs from
their instantiation relationships.  For example, by using specialization 
hierarchies, we can retrieve all the instances of the same parent.  This 
manipulation is frequently used in investigating the progress status of
work.

{\em General} operation, noted as $\wfbgeneral(W, \theta)$ is defined as
\[
 \wfbgeneral (W, \theta) \mydefi \{W' | W'\theta = W\}.
\]
This operation obtains a set of WFTs each of which is a parent of $W$
over the partial ordering $\sqsubseteq$.

{\em Special} operation, noted as $\wfbspecial(W, \theta)$, is defined as 
\[
 \wfbspecial(W, \theta) \mydefi \{W' | W' = W\theta\}.
\]
This operation obtains a set of WFTs each of which is a child of a WFT
$W$ over the partial ordering $\sqsubseteq$.

We also define the transitive closure of $\wfbgeneral$ and
$\wfbspecial$, noted as $\wfbancestor$ and $\wfbdescendant$,
respectively.  These operation obtain all the ancestors and all the
descendants on the specialization hierarchies, respectively.

\section{Operations Handling WFTs}
\label{sec:changewfop}

%Ordinary set operations can be used as the operators on WFTs, though we
%don't mention about the their definitions in this paper.

\subsection{Grouping, Ungrouping}

First we define an operation for grouping multiple activity objects into 
one activity object, and its reverse operation.

For a closed WFT $W = \{a_1, \cdots, a_n\}$, {\em grouping} operation,
noted as $\wfbgrouping W$, returns an activity object
\[
 a' = (\bigcup^n_{i=1}I_i-\bigcup^n_{i=1}O_i,
 \bigcup^n_{i=1}O_i-\bigcup^n_{i=1}I_i, \bigcup^n_{i=1}P_i, W), 
\]
and lets $W' = (W' - W) \cup \{a\}$ for all WFTs $W'. W \subseteq W'$.
Intuitively this is a ``grouping'' operation.  It regards a WFT
$W$ as a sequence of one work, and creates an activity object
including $W$ as subworks.

Next the reverse operation of grouping, {\em flattening} operation,
noted as $\wfbflatten a$, is defined.
Let $a_1 = (I_1, O_1, P_1, S_1)$.  For all activity objects $a_1$ where
$a = (I, O, P, S) \in S_1$, this operation lets $S_1 = S_1 \cup S -
\{a\}$ if $S \neq \emptyset$.  If $S = \emptyset$, it does nothing.

\subsection{Split, Concatenation}
{\em Split} operations, noted as $\wfbsplit(a, W)$, is defined as
follows.  For an activity object $a = (I, O, P, S)$ and a closed and
consistent WFT $W = \{a_1, \cdots, a_n\}$ where $W \subseteq S$, this
operation returns a WFT $W' = \{a'_1, a'_2\}$ where:
\begin{eqnarray*}
 a'_1 & = & (\bigcup^n_{i=1}I_i-\bigcup^n_{i=1}O_i,
  \bigcup^n_{i=1}O_i-\bigcup^n_{i=1}I_i, P, W)\\
 a'_2 & = & ((I - I'_1) \cup O'_1, (O - O'_1) \cup I'_1, P, S-W)
\end{eqnarray*}

Intuitively this operation splits a WFT $S$ from a specified WFT $W$.
In general, the obtained WFT $W'$ is not always acyclic in
$\Longrightarrow$.  That is, two horizontal flows, $a'_1 \Longrightarrow
a'_2$ and $a'_2 \Longrightarrow a'_1$, hold simultaneously.  To make
$W'$ to be acyclic in $\Longrightarrow$, one of these two conditions
must be satisfied: $I'_1 - O'_1 \subseteq I$ or $O'_1 - I'_1 \subseteq
O$.

{\em Concatenation} operation, noted as $\wfbconcat(a_1, a_2)$, is
closely related with split operation.
If neither $a_1 \Longrightarrow a_2$ nor $a_2 \Longrightarrow
a_1$ are satisfied, this operation lets $I_2 = I_2 \cup O_1$.  If
one of the two conditions are satisfied, this operation does
nothing.

Intuitively it connects two activity objects $a_1, a_2$ by a
horizontal flow.  From the definition of horizontal flow, the
outputs of $a_1$ must be added to the inputs of $a_2$ when no
horizontal flow is found between $a_1$ and $a_2$.

\section{Operations Handling Activity Objects}
\label{sec:changeaoop}

Some kinds of changes on WFTs, such as changing flows or adding new
activity objects, need operations on activity objects in the WFTs
directly.  Moreover, operations handling WFTs shown in the previous
section are implemented using operations on activity objects.  For these 
reasons, operations handling activity objects directly are necessary for 
WFB.  

In addition to the conventional set operations or the operations in
relational algebra, we provide these operations listed below for
handling activity objects:
\begin{itemize}
 \item $\operator{new}(a, I, O, P, S)$\\
       It creates a new activity object.
 \item $\operator{destroy}(a)$\\
       It destroys a specified activity object $a$.
 \item $\operator{foreach}\; a\; \operator{in}\; s\; \operator{do}\;
       Operator$\\
       It executes an operation WFT $Operator$ for each activity object
       $a$ in a set $s$.  $Operator$ can be any operator on an activity
       object.  This operator is useful when the maintainer wants to do
       the same operations on the number of activity objects.  For
       example, if yokota's works are all transferred to kunishima, the
       maintainer will do the following operation:
       \[
	\operator{foreach}\; a\; \operator{in}\;
       \selectao_{yokota\in P} W_{all} \; \operator{do}
       \;\; a.P := \{kunishima\};
       \]
\end{itemize}

However, these operations may cause integrity violations of WFTs.  For
example, if the operation $\operator{destroy}(\mbox{dispatch-1})$ is
applied on the workflow ``review-process'' in Figure \ref{f:wft},
``review-process'' will be no more than a workflow because it is not
closed and not connected.  Hence the operations shown in this section
are mainly for the maintainers of workflow base.

\section{Operations on Activity Objects With Plugins}

{\em Plug-in} operation, $\wfbplugin$, is provided for adding plugins
such as time plugin or resource plugin to activity objects.  It is
defined as: 
\[
 a \wfbplugin L \mydefi a \Join L 
\]
 where $a$ is an activity object and $L$ is a plugin.  As shown in this
definition, plug-in operation on an activity object is in actual same as
natural join operations in relational algebra\footnote{The notations of
the operators in relational algebra are based on
\cite{ullman:database88:1}.}.

Plug-in operation $\wfbplugin$ on a WFT is defined as:
\[
 W \wfbplugin L = \{a_1 \wfbplugin L, \cdots, a_n \wfbplugin L\}
\]
where $W=\{a_1, \cdots, a_n\}$.

{\em Plug-out} operation, $\wfbplugout$, is defined as a reverse operator of
$\wfbplugin$ such as:
\[
 \wfbplugout_{L} a_1 \mydefi (I, O, P, S)
\]
where $a_1 = (I, O, P, S, L)$.  Plug-out operation is a special case of 
projection operation in relational algebra.

Plug-out operation on a WFT can be defined as same as the plug-in
operator on a WFT:
\[
 \wfbplugout_{L} W = \{\wfbplugout_{L} a_1, \cdots, \wfbplugout_{L} a_n\} 
\]
where $W = \{a_1, \cdots, a_n\}$.

Plug-in operation and plug-out operation can be applied into activity
objects already plugged in by another plugins, with no modification of
the definitions above.

\section{Examples on Applying Operations}

We show some examples of applying the operations on workflow-base.

\subsection{Retrievals}

Consider a situation doing a review process based on the workflow in
Figure \ref{f:wft}.  The chair wants to investigate the progress of the
review process of a submitted paper ``A100''.  He gets the workflow
of the process by applying this operation:
\[
 \getworkflow \selectao_{A100 \in I} W_{all};
\]

Then he finds the review process is delayed at reviewer {\it Taro}.  He
wants to know all the works {\it Taro} is doing.  {\it Taro}'s all
activity objects can be obtained by applying the following:
\[
 W_{taro} := \selectao_{Taro \in P} W_{all};
\]
The precise of each workflows can be obtained by:
\[
 W_{taroPrecise} := \wfbshow W_{taro};
\]

Consider another query.  A committee member {\it jiro} wants to
investigate that other committee members have asked {\it taro} to review
submitted papers.  Because $\mbox{review}_{11}, \cdots,
\mbox{review}_{nj_k}$ in Figure \ref{f:box} are children of a WFT ``review''
over the partial ordering $\sqsubseteq$, this query is described as the
following:
\[
 \selectao_{taro \in P} \selectwft_{\wfi(review)} W_{all};
\]

\subsection{Updates of Workflows}

Figure \ref{fig:registration} shows a workflow of a registration process of 
the same conference.
\begin{figure}
\begin{center}
\begin{tabular}{lcl}
 registration-process & = & \{registration, send-registration\}\\
 registration & = & (entry-sheet, participant-name, secretary, \\
              &   & \{register into the participants list.\})\\
 send-registration & = & (participant-name, acceptance-sheet, \\
                   &   & secretary, \{\})
\end{tabular}
\end{center}
\caption{WFT of a Registration Process}
\label{fig:registration}
\end{figure}

``send-registration'' in Figure \ref{fig:registration} and
``send'' in Figure \ref{f:wft} are almost the same process
in the sense that a secretary sends something to someone.  For this
reason, these two processes can be joined into one process
``send-object'' by applying the following operations: 
\[
 \begin{array}{l}
 \operator{new}(\mbox{send-object}, \mbox{address}, \mbox{objects},
  \mbox{secretary}, \{\});\\
 \wfbsplit(W_1, \{\mbox{send-review-report}\});\\
 \wfbsplit(W_2, \{\mbox{send-registration}\});\\
 \wfbgeneral(\{\mbox{send}\},\\
  \{\mbox{address/participants-name}, \mbox{objects/acceptance-sheet}\});\\
 \wfbgeneral(\{\mbox{send-registration}\},\\
 \{\mbox{address/all-review-reports}, \mbox{objects/review-letter}\});
 \end{array}
\]

\section{Summary}

Operations to a WFB are defined as an extension of ordinary relational
algebra.  These operations get a set of WFTs or a set of activity
objects from WFTs in WFB\@.  They do not always keep the closure property
of relational algebra.  Some operations are provided in order to let a
set of activity objects into a WFT\@.  The combination of these operations
with others makes the closure property to be kept.

When using these operations, users can organize new workflows from the
workflows in WFB\@.  For example, the following functions are available
using the operations: views over workflows, such as to-do lists;
investigation of progress; workflow reorganization; reuse of workflows.

The operations are classified into three categories: query operations,
operations handling WFTs, and operations handling activity objects.
